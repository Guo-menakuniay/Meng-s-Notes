\section{电磁场的矢势和标势}
    \subsection{用势描述电磁场}
        我们对电磁场也引入矢势和标势,我们从之前引入的磁场矢势入手\[\nabla \times \boldsymbol{B} \]在一般的情况下,磁场依然保持无源场的性质,但是在电磁波中电场同时具有有源和有旋场的性质,这意味着电场关系式中也会出现磁矢势
            \begin{equation}
                \nabla \times (\boldsymbol{E} +\frac{\partial \boldsymbol{A}}{\partial t})=0
            \end{equation}
        上面的式子可以用一个标势的负梯度描述:
            \begin{equation}
                \boldsymbol{E} + \frac{\partial \boldsymbol{A}}{\partial t} = -\nabla \varphi
            \end{equation}
        因而得到电场表达式\footnote{这一项是满足规范变换的,即\textbf{势做规范变换时,所有物理量和物理规律都应该保持不变,这种不变性称为规范不变性。}}
            \begin{equation}
                \boldsymbol{E} = -\frac{\partial \boldsymbol{A}}{\partial t} -\nabla \varphi
            \end{equation}
    \subsection{达朗贝尔方程}
        我们用麦克斯韦方程和上面的电磁场势进行推导,首先有关系
        \begin{equation*}
            \begin{gathered}
                \frac{\partial \boldsymbol{D}}{\partial t} = \varepsilon ( \frac{\partial \boldsymbol{E}}{\partial t}) = \varepsilon  \frac{\partial }{\partial t}(-\nabla \varphi -  \frac{\partial \boldsymbol{A}}{\partial t}) \\
                \nabla \times \boldsymbol{H} =  \frac{\partial \boldsymbol{D}}{\partial t} + \boldsymbol{J}
            \end{gathered}
        \end{equation*}
        可以得到
        \begin{equation}
            \nabla \times(\nabla \times \boldsymbol{A})= \mu_{0} \boldsymbol{J}-\mu_{0} \varepsilon_{0} \frac{\partial}{\partial t} \nabla \varphi-\mu_{0} \varepsilon_{0} \frac{\partial^{2} \boldsymbol{A}}{\partial t^{2}}-\nabla^{2} \varphi-\frac{\partial}{\partial t} \nabla \cdot \boldsymbol{A}=\frac{\rho}{\varepsilon_{0}}
        \end{equation}
        将真空光速的定义式代入进行化简可以得到
        \begin{equation}
            \begin{gathered}
            \nabla^{2} \boldsymbol{A}-\frac{1}{c^{2}} \frac{\partial^{2} \boldsymbol{A}}{\partial t^{2}}-\nabla\left(\nabla \cdot \boldsymbol{A}+\frac{1}{c^{2}} \frac{\partial \varphi}{\partial t}\right)=-\mu_{0} \boldsymbol{J} \\
            \nabla^{2} \varphi+\frac{\partial}{\partial t} \nabla \cdot \boldsymbol{A}=-\frac{\rho}{\varepsilon_{n}}
            \end{gathered}
        \end{equation}
        为保证电磁场的矢势和标势的形式是对称的,在这里不采用库伦规范,我们采用洛伦兹规范:
        \begin{equation}
            \nabla \cdot \boldsymbol{A} + \frac{1}{c^2} \frac{\partial \varphi}{\partial t}=0
        \end{equation}
        则达朗贝尔方程即为
        \begin{equation}
            \label{eq.5_7}
            \boxed{\begin{gathered}
            \nabla^{2} \boldsymbol{A}-\frac{1}{c^{2}} \frac{\partial^{2} \boldsymbol{A}}{\partial t^{2}}=-\mu_{0} J \\
            \nabla^{2} \varphi-\frac{1}{c^{2}} \frac{\partial^{2} \varphi}{\partial t^{2}}=-\frac{\rho}{\varepsilon_{0}} 
            \end{gathered}}
        \end{equation}
        
        方程在形式上可以看出,电荷产生标势波动,电流产生矢势波动。离开电荷电流分布区域之后,两者都以波动的形式在空间中传播。
\section{推迟势}
    由达朗贝尔导出,对于静态情形\footnote{这一节采用了格里菲斯教材的思路,回避了对达朗贝尔方程求解的具体过程,求解过程部分详细可参见郭硕鸿先生教材}
    \begin{equation}
        \begin{gathered}
            \nabla^2 \boldsymbol{\varphi} = -\frac{\rho}{\varepsilon_0} \\
            \nabla^2 \boldsymbol{\boldsymbol{A}} = -\mu_0 \boldsymbol{J}
        \end{gathered}
    \end{equation}
    有两个熟知的解
    \begin{equation}
        \begin{gathered}
            V(\boldsymbol{r}) = \frac{1}{4 \pi \varepsilon_0} \int \frac{\rho(\boldsymbol{r}^\prime)}{r}dV \\
            V(\boldsymbol{r}) = \frac{\mu_0}{4 \pi} \int \frac{\rho(\boldsymbol{J}^\prime)}{r}dV
        \end{gathered}
    \end{equation}
    其中r为源点到场点的距离。而在非静止的情况下,由于电磁波益光速传播,当一个事件从源点传到场点的时候,中间应该经过了一段推迟的时间$r/c$,或者我们可以理解成在t时刻场点受到的作用实际上来自$t-r/c$时刻的源点,我们将它称为推迟势。

    我们直接由静态情况下的解猜测推迟势的解为
    \begin{equation}
        \boxed{V(\boldsymbol{r}, t)=\frac{1}{4 \pi \varepsilon_{0}} \int \frac{\rho\left(\boldsymbol{r}^{\prime}, t_{\mathrm{r}}\right)}{r} \mathrm{~d} \tau^{\prime}, \quad \boldsymbol{A}(\boldsymbol{r}, t)=\frac{\mu_{0}}{4 \pi} \int \frac{\boldsymbol{J}\left(\boldsymbol{r}^{\prime}, t_{r}\right)}{r} \mathrm{~d} \tau^{\prime}}
    \end{equation}
    其中$t_r=t-r/c$
    
    下面我们来验证这一假设解确实是达朗贝尔方程的解\footnote{笔者更喜欢把大段的数学推导放在脚注以保持物理图像的连贯性。在计算 $V(\boldsymbol{r}, t)$ 的拉普拉斯算子时, 特别要注意的是积分在两处依赖 $r$ : 显含在分母中的 $\left(r=\left|\boldsymbol{r}-\boldsymbol{r}^{\prime}\right|\right)$ 和隐含在分子中的 $t_{\mathrm{r}} \equiv t-r / c$$$\nabla V=\frac{1}{4 \pi \varepsilon_{0}} \int\left[(\nabla \rho) \frac{1}{2}+\rho \nabla\left(\frac{1}{r}\right)\right] \mathrm{d} \tau^{\prime}$$和$$\nabla \rho=\dot{\rho} \nabla t_{r}=-\frac{1}{c} \dot{\rho} \nabla r$$(式中的点表示对时间的微分)现在有 $\nabla r=\hat{r}$ 和 $\nabla(1 / r)=-\hat{r} / r^{2}$ , 故$$\nabla V=\frac{1}{4 \pi \varepsilon_{0}} \int\left[-\frac{\dot{\rho}}{c} \frac{\hat{r}}{r}-\rho \frac{\hat{r}}{r}\right] \mathrm{d} \tau^{\prime}$$取散度,$$\nabla^{2} V=\frac{1}{4 \pi \varepsilon_{0}} \int\left\{-\frac{1}{c}\left[\frac{\hat{r}}{r} \cdot(\nabla \dot{\rho})+\dot{\rho} \nabla \cdot\left(\frac{\hat{r}}{r}\right)\right]-\left[\frac{\hat{r}}{r^{2}} \cdot(\nabla \rho)+\rho \nabla \cdot\left(\frac{\hat{r}}{r^{2}}\right)\right]\right\} \mathrm{d} \tau^{\prime}$$但$$\nabla \dot{\rho}=-\frac{1}{c} \ddot{\rho} \nabla r=-\frac{1}{c} \ddot{\rho} \hat{r}$$这同式 (10.21) 中的一样, 并且$$\nabla \cdot\left(\frac{\hat{r}}{r}\right)=\frac{1}{r^{2}}$$和$$\nabla \cdot\left(\frac{\hat{r}}{r^{2}}\right)=4 \pi \delta^{3}(r)$$故有\[\nabla^{2} V=\frac{1}{4 \pi \varepsilon_{0}} \int\left[\frac{1}{c^{2}} \frac{\ddot{\rho}}{2}-4 \pi \rho \delta^{3}(\boldsymbol{r})\right] \mathrm{d} \tau^{\prime}=\frac{1}{c^{2}} \frac{\partial^{2} V}{\partial t^{2}}-\frac{1}{\varepsilon_{0}} \rho(\boldsymbol{r}, t)\]证明了推迟势确实是达朗贝尔方程的解}

    根据推迟势公式,当$\rho$和$\boldsymbol{J}$给定之后就可以得到势,而根据下式可以算出电磁场
    \begin{equation}
        \begin{gathered}
            \boldsymbol{B} = \nabla \times \boldsymbol{A} \\
            \boldsymbol{E} = - \frac{\partial \boldsymbol{A}}{\partial t} - \nabla \varphi
        \end{gathered}
    \end{equation}
    事实上场会反作用于电荷电流分布,故激发区的电荷电流不能随意分布,我们之后为详细介绍这一点。
\section{几种常见的辐射形式}
    \footnote{本段内容来自格里菲斯版电动力学内容}我们已知球壳的面积是$4 \pi r^2$,即能流密度的减少量不能快于$1/r^2$:因为如果它以$1/r^3$的速度减少,则无限远处能流密度的积分应该是零——这是不符合常理的,因为能流密度在任意远的地方积分都应该相等。下面我们以静止的电荷为例,由库伦定律可知电场以$1/r^2$减少,由毕奥-萨法尔定律可知磁场也以$1/r^2$(或更快)的方式减少,则能流密度以$1/r^4$的速度减少,故\textbf{静止的源不产生辐射}。但与时间相关的场,包含一些以$1/r$变化的项,正是这些项导致了电磁辐射。
    \subsection{计算辐射场的一般公式}
        \textbf{计算辐射场的基本是推迟势公式}

        对于推迟势公式,以矢势为例\[\quad \boldsymbol{A}(\boldsymbol{r}, t)=\frac{\mu_{0}}{4 \pi} \int \frac{\boldsymbol{J}\left(\boldsymbol{r}^{\prime}, t_{r}\right)}{r} \mathrm{~d} \tau^{\prime}\]计算得到电磁场的矢势之后,就可以经过下式计算电磁场量
        \begin{equation}
            \boxed{\begin{gathered}
                \boldsymbol{B} = \nabla \times \boldsymbol{A} \\
                \boldsymbol{E} = \frac{i c}{k} \nabla \times \boldsymbol{B}
            \end{gathered}}
        \end{equation}

        在矢势公式中有三个距离的线度需要纳入考虑:电荷分布区域的线度$l$,波长$\lambda$,电荷到场点的距离$r$。本节中我们都研究的是小区域内电荷产生的辐射,即满足\[l \ll \lambda, \quad l \ll r\]而对于后两个量同样可以按照线度的大小关系分为:
        \begin{enumerate}[(1).]
            \item 近区 ($r \ll \lambda$),在近区内电磁场保持着恒定场的主要特点
            \item 感应区 ($r ~ \lambda$),性质介于两者之间
            \item 辐射区 ($r \gg \lambda$),在辐射区内电磁场变为横向的辐射场
        \end{enumerate}
        
        我们对矢势中的$r$展开得到\footnote{Unfinished.啊啊啊啊复习不完了复习不完了,这里跳过了展开步骤,以后有空补上}
        \begin{equation}
            \label{eq.5_13}
            \boldsymbol{A}(x) = \frac{\mu_0 e^{ikR}}{4 \pi R}\int_V \boldsymbol{J}(x^\prime)(1-ik\mathbf{e}_R \cdot x^\prime + \cdots)dV^\prime
        \end{equation}
    \subsection{电偶极辐射}
        电偶极辐射来源于\ref{eq.5_13}展开式中的第一项
        \begin{equation}
            \boldsymbol{A} = \frac{\mu_0 e^{ikR}}{4 \pi R}\int_V \boldsymbol{J}(x^\prime) dV^\prime
        \end{equation}
        其中积分项就是电偶极矩对时间的偏导数\[\int_V \boldsymbol{J}(x^\prime)dV^\prime= \sum q \boldsymbol{v}=\frac{d}{d t}\sum q \boldsymbol{x}=\frac{d \boldsymbol{p}}{d t}=\boldsymbol{\dot{p}}\]可以直接写出电偶极辐射的公式
        \begin{equation}
            \boldsymbol{A}(x) = \frac{\mu_0 e^{i k R}}{4 \pi R}\boldsymbol{\dot{p}}
        \end{equation}
        根据势的公式计算电磁场的时候,因为分母中的R是我们专门展开需要的,故$\nabla$项就只需要作用在相因子上,意味着对于以下算符可以代换为
        \begin{equation}
            \label{eq.5_17}
            \boxed{\begin{gathered}
                \nabla \to i k \boldsymbol{e}_R \\
                \frac{\partial }{\partial t} \to -i \omega 
            \end{gathered}}
        \end{equation}

        可以计算得到辐射场\footnote{我们都将结果写作只含$\varepsilon_0$和k的形式,用关系代换掉$\mu_0$}
        \begin{equation}
            \begin{aligned}
            \boldsymbol{B} &=\boldsymbol{\nabla} \times \boldsymbol{A}=\frac{\mathrm{i} \mu_{0} k}{4 \pi R} \mathrm{e}^{i k R} \boldsymbol{e}_{R} \times \dot{\boldsymbol{p}}=\frac{1}{4 \pi \varepsilon_{0} c^{3} R} \mathrm{e}^{\mathrm{i} k R} \ddot{\boldsymbol{p}} \times \boldsymbol{e}_{R} \\
            \boldsymbol{E} &=\frac{\mathrm{i} c}{k} \boldsymbol{\nabla} \times \boldsymbol{B}=c \boldsymbol{B} \times \boldsymbol{e}_{R}=\frac{\mathrm{e}^{\mathrm{i} k R}}{4 \pi \varepsilon_{0} c^{2} R}\left(\ddot{\boldsymbol{p}} \times \boldsymbol{e}_{R}\right) \times \boldsymbol{e}_{R}
            \end{aligned}
        \end{equation}
        在这种情况下,辐射区的电磁场$~1/R$,能流密度$~1/R^2$,对球面积分之后总功率与球的半径无关,保证了电磁能量可以传播到任意远处,和之前的理论吻合。
        
        \subsubsection{辐射能流、功率与角分布}
            电偶极辐射的平均能流密度可以计算得
            \begin{equation}
                \begin{aligned}
                \overline{\boldsymbol{S}} &=\frac{1}{2} \operatorname{Re}\left(\boldsymbol{E}^{*} \times \boldsymbol{H}\right)=\frac{c}{2 \mu_{0}} \operatorname{Re}\left[\left(\boldsymbol{B}^{*} \times \boldsymbol{e}_{R}\right) \times \boldsymbol{B}\right] \\
                &=\frac{c}{2 \mu_{0}}|\boldsymbol{B}|^{2} \boldsymbol{e}_{R}=\frac{|\ddot{p}|^{2}}{32 \pi^{2} \varepsilon_{0} c^{3} R^{2}} \sin ^{2} \theta \boldsymbol{e}_{R} 
                \end{aligned}
                \end{equation}
            式中$\sin^2$项即体现了辐射的角分布,可以用平均能流密度对球面积分可以得到总辐射功率\footnote{可以用整个球的立体角定义来记忆立体角的积分公式\[\frac{d S}{d\Omega}=\frac{S}{\Omega}=\frac{4 \pi R^2}{4 \pi}=R^2\]}
            \begin{equation}
                \begin{aligned}
                P &=\oint|\overline{\boldsymbol{S}}| R^{2} \mathrm{~d} \Omega \\
                &=\frac{|\ddot{\boldsymbol{p}}|^{2}}{32 \pi^{2} \varepsilon_{0} c^{3}} \oint \sin ^{2} \theta \mathrm{d} \Omega \\
                &=\frac{1}{4 \pi \varepsilon_{0}} \frac{|\ddot{\boldsymbol{p}}|^{2}}{3 c^{3}}
                \end{aligned}
            \end{equation}
            
            结合关系式\ref{eq.5_17}可以得到,总辐射功率正比于$\omega^4$,当电偶极子振荡频率增高的时候辐射功率迅速增加。
        
        \subsubsection{短天线辐射和辐射电阻}\footnote{Unfinished:留坑}
    \subsection{高频电流分布的磁偶极矩和电四极矩}
        \textcolor[RGB]{143,143,143}{笔者还没搞明白对称张量分离的办法,格里菲斯上的推导办法还没来得及看,这里先直接给出公式和结论}
        
        现在我们计算矢势展开的第二项
        \begin{equation}
            \boldsymbol{A} = \frac{- i k \mu_0 e^{ikR}}{4 \pi R}\int_V \boldsymbol{J}(x^\prime)(\mathrm{e}_R \cdot x^\prime) dV^\prime
        \end{equation}
        直接给出结果
        \begin{equation}
            \boldsymbol{A}(\boldsymbol{x})=-\frac{\mathrm{i} k \mu_{0} \mathrm{e}^{\mathrm{i} k R}}{4 \pi R}\left(-\boldsymbol{e}_{R} \times m+\frac{1}{6} \boldsymbol{e}_{R} \cdot \dot{\mathscr{D}_{ij}}\right)
        \end{equation}
        其中$\mathscr{D_{ij}}$为体系的电四极矩\[\mathscr{D}_{ij}=\sum3q\boldsymbol{x^\prime}\boldsymbol{x^\prime}\]

        式中括号第一项为磁偶极辐射势,第二项为电四极辐射势。他们在电磁矢势的同一级展开式中体现。
    \subsection{磁偶极辐射}
        我们考虑磁偶极辐射的性质,即
        \begin{equation}
            \boldsymbol{A}(x) = \frac{ik \mu_0 e^{ik R}}{4 \pi R}\boldsymbol{e}_R \times \boldsymbol{m}
        \end{equation}
        计算电磁场\footnote{这里和之前一样,都使用了\[( \frac{\partial }{\partial t})^2=(-i \omega)^2=-\omega^2\]结合$k$的定义来化简,其中负号吸收进了矢积改变了乘积方向}
        \begin{equation}
            \begin{aligned}
            \boldsymbol{B} &=\boldsymbol{\nabla} \times \boldsymbol{A}=\mathrm{i} k \boldsymbol{e}_{R} \times \boldsymbol{A} \\
            &=k^{2} \frac{\mu_{0} \mathrm{e}^{\mathrm{i} k R}}{4 \pi R}\left(\boldsymbol{e}_{R} \times \boldsymbol{m}\right) \times \boldsymbol{e}_{R} \\
            &=\frac{\mu_{0} \mathrm{e}^{\mathrm{i} k R}}{4 \pi c^{2} R}\left(\ddot{\boldsymbol{m}} \times \boldsymbol{e}_{R}\right) \times \boldsymbol{e}_{R} \\
            \boldsymbol{E} &=\frac{i c}{k} \nabla \times \boldsymbol{B} =-\frac{\mu_{0} \mathrm{e}^{\mathrm{i} k R}}{4 \pi c R}\left(\ddot{\boldsymbol{m}} \times \boldsymbol{e}_{R}\right)
            \end{aligned}
        \end{equation}

        我们知道电场和磁场在真空中有良好的对称性,我们比较上式和\ref{eq.5_17}可以发现只需以下代换就可以使两者等价
        \begin{equation}
            \begin{gathered}
                \boldsymbol{p} \to \frac{\boldsymbol{m}}{c} \\
                \boldsymbol{E} \to c \boldsymbol{B} \\
                c \boldsymbol{B} \to - \boldsymbol{E}
            \end{gathered}
        \end{equation}
        在对称性的体现上,即得在自由空间中, 麦氏方程组对变换 $\boldsymbol{E} \rightarrow c \boldsymbol{B}, c \boldsymbol{B} \rightarrow-\boldsymbol{E}$ 是对称的。若电磁场 $\boldsymbol{E}(\boldsymbol{x}, \boldsymbol{t}), \boldsymbol{B}(\boldsymbol{x}, \boldsymbol{t})$ 是麦克斯韦方程组的解,则代换后的电磁场也是麦克斯韦方程组的解。

        可以用代换直接得到磁偶极辐射的能流密度
        \begin{equation}
            \boldsymbol{S}=\frac{\mu_{0} \omega^{4}|m|^{2}}{32 \pi^{2} c^{3} R^{2}} \sin ^{2} \theta \boldsymbol{e}_{R}
        \end{equation}
        总辐射功率
        \begin{equation}
            P =\frac{\mu_0 \omega^4 |\boldsymbol{m}|^2}{12 \pi c^3}
        \end{equation}
    \subsection{电四极辐射}
        对于第二个展开项
        \begin{equation}
            \boldsymbol{A}(\boldsymbol{x})=-\frac{\mathrm{i} k \mu_{0} \mathrm{e}^{\mathrm{i} k R}}{24 \pi R}\boldsymbol{e}_{R} \cdot \dot{\mathscr{D}_{ij}}
        \end{equation}
        本节也直接给出结果,定义矢量\[\mathscr{D}_{ij}^\prime = \boldsymbol{e}_R \cdot \mathscr{D}_{ij}\]则辐射区的电磁场为
        \begin{equation}
            \begin{gathered}
                \boldsymbol{B}=\mathrm{i} k \boldsymbol{e}_{R} \times \boldsymbol{A}=\frac{\mathrm{e}^{\mathrm{i} k R}}{24 \pi \varepsilon_{0} c^{4} R} \dddot{\mathscr{D}_{ij}^\prime} \times \boldsymbol{e}_{R} \\
                \boldsymbol{E}=c \boldsymbol{B} \times \boldsymbol{e}_{R}=\frac{\mathrm{e}^{\mathrm{i} k R}}{24 \pi \varepsilon_{0} c^{3} R}\left(\dddot{\mathscr{D}_{ij}^\prime} \times \boldsymbol{e}_{R}\right) \times \boldsymbol{e}_{R}
            \end{gathered}
        \end{equation}
        辐射平均能流密度暂且不表。直接给出结论:电四极辐射和磁偶极辐射的功率是同数量级的。
\section{电磁波的衍射}
    \subsection{基尔霍夫公式}
        电磁场任一分量都满足亥姆霍兹方程,即\[\nabla^2 \psi +k^2 \psi =0\]如果我们忽略其他电磁场分量的影响,把$\psi$看作一个标量场,用边界上的$\psi$$ \frac{\partial \psi}{\partial n}$表出边界条件,这种理论就是标量衍射理论。

        可以用格林函数变形得到
        \begin{equation}
            \begin{aligned}
            \phi(\boldsymbol{x}) &=-\frac{1}{4 \pi} \oint_{S}\left[\psi\left(\boldsymbol{x}^{\prime}\right) \boldsymbol{\nabla}^{\prime} \frac{\mathrm{e}^{\mathrm{i} k r}}{r}-\frac{\mathrm{e}^{\mathrm{i} k r}}{r} \nabla^{\prime} \psi\left(\boldsymbol{x}^{\prime}\right)\right] \cdot \mathrm{d} \boldsymbol{S}^{\prime} \\
            &=-\frac{1}{4 \pi} \oint_{S} \frac{\mathrm{e}^{\mathrm{i} k r}}{r} \boldsymbol{e}_{\mathrm{n}} \cdot\left[\nabla^{\prime} \psi+\left(\mathrm{i} k-\frac{1}{r}\right) \frac{\boldsymbol{r}}{r} \psi\right] \mathrm{d} S^{\prime}
            \end{aligned}
            \end{equation}
        上式为\textbf{基尔霍夫公式},是惠更斯原理的数学表示。
    \subsection{夫琅和费衍射}
        碍于记录的时间原因,这里直接给出结果\footnote{Unfinished}
        \begin{equation}
            \begin{aligned}
            \psi(\boldsymbol{x}) &=-\frac{\mathrm{i} \psi_{0} \mathrm{e}^{\mathrm{i} k R}}{4 \pi R} \int_{S_{0}} \mathrm{e}^{\mathrm{i}\left(\boldsymbol{k}_{1}-\boldsymbol{k}_{2}\right) \cdot x^{\prime}}\left(\boldsymbol{k}_{1}+\boldsymbol{k}_{2}\right) \cdot \boldsymbol{e}_{\mathrm{n}} \mathrm{d} S^{\prime} \\
            &=-\frac{\mathrm{i} \psi_{0} \mathrm{e}^{\mathrm{i} k R}}{4 \pi R} \int_{S_{0}} \mathrm{e}^{\mathrm{i}\left(\boldsymbol{k}_{1}-\boldsymbol{k}_{2}\right) \cdot x^{\prime}}\left(\cos \theta_{1}+\cos \theta_{2}\right) \mathrm{d} S^{\prime}
            \end{aligned}
            \end{equation}
        式中 $\theta_{1}$ 为入射波矢 $\boldsymbol{k}_{1}$ 与法线方向 $\boldsymbol{e}_{\mathrm{n}}$ 的夹角, $\theta_{2}$为衍射波矢 $\boldsymbol{k}_{2}$ 与 $\boldsymbol{e}_{\mathrm{n}}$ 的夹角. $\cos \theta_{1}+\cos \theta_{2}$ 称为倾斜因子。
\section{电磁场的动量}
    \subsection{麦克斯韦应力张量}
        我们已知单位体积电荷收到的力可以表示为
        \begin{equation}
            \boldsymbol{f} = \rho \boldsymbol{E} + \boldsymbol{J} \times \boldsymbol{B}
        \end{equation}
        用麦克斯韦方程组展开并且加入一项$(\nabla \cdot \boldsymbol{B})\boldsymbol{B}$以保持结果中电场和磁场形式的对称(因为$\nabla \cdot \boldsymbol{B}$等于零,对结果没有任何影响),这样就有
        \begin{equation}
            \boldsymbol{f}=\varepsilon_{0}[(\nabla \cdot \boldsymbol{E}) \boldsymbol{E}+(\boldsymbol{E} \cdot \nabla) \boldsymbol{E}]+\frac{1}{\mu_{0}}[(\nabla \cdot \boldsymbol{B}) \boldsymbol{B}+(\boldsymbol{B} \cdot \nabla) \boldsymbol{B}]- \frac{1}{2} \nabla\left(\varepsilon_{0} E^{2}+\frac{1}{\mu_{0}} B^{2}\right)-\varepsilon_{0} \frac{\partial}{\partial t}(\boldsymbol{E} \times \boldsymbol{B})
        \end{equation}
        引入\textbf{麦克斯韦应力张量\footnote{一般二阶张量可以用双箭头表示\[\overleftrightarrow{T}\]}}可以大幅简化上式的形式\footnote{其是一个对称张量,或者直接用一个矩阵表示为\[T_{i j}=\left(\begin{array}{ccc}
            \epsilon_{0}\left(E_{x}^{2}-E^{2} / 2\right)+\frac{1}{\mu_{0}}\left(B_{x}^{2}-B^{2} / 2\right) & \epsilon_{0} E_{x} E_{y}+\frac{1}{\mu_{0}}\left(B_{x} B_{y}\right) & \epsilon_{0} E_{x} E_{z}+\frac{1}{\mu_{0}}\left(B_{x} B_{z}\right) \\
            \epsilon_{0} E_{x} E_{y}+\frac{1}{\mu_{0}}\left(B_{x} B_{y}\right) & \epsilon_{0}\left(E_{y}^{2}-E^{2} / 2\right)+\frac{1}{\mu_{0}}\left(B_{y}^{2}-B^{2} / 2\right) & \epsilon_{0} E_{y} E_{z}+\frac{1}{\mu_{0}}\left(B_{y} B_{z}\right) \\
            \epsilon_{0} E_{x} E_{z}+\frac{1}{\mu_{0}}\left(B_{x} B_{z}\right) & \epsilon_{0} E_{y} E_{z}+\frac{1}{\mu_{0}}\left(B_{y} B_{z}\right) & \epsilon_{0}\left(E_{z}^{2}-E^{2} / 2\right)+\frac{1}{\mu_{0}}\left(B_{z}^{2}-B^{2} / 2\right)
            \end{array}\right)\]张量中的ij元素诠释为,朝著i-轴方向,施加于j-轴的垂直平面,单位面积的作用力\\ \textcolor[RGB]{143,143,143}{Unfinished.这里矩阵的间距太窄不够美观,有时间修复}}:
        \begin{equation}
            T_{i j} \equiv \varepsilon_{0}\left(E_{i} E_{j}-\frac{1}{2} \delta_{i j} E^{2}\right)+\frac{1}{\mu_{0}}\left(B_{i} B_{j}-\frac{1}{2} \delta_{i j} B^{2}\right)
        \end{equation}
        麦克斯韦应力张量矢量写法和标量写法之间有关系
        \begin{equation}
            (a \cdot \overleftrightarrow{T})_{j}=\sum_{i=x, y, z} a_{i} T_{i j}
        \end{equation}
        特别的,对于$\overleftrightarrow{T}$散度的第j个分量可以表示为
        \begin{equation}
            (\nabla \cdot \stackrel{\leftrightarrow}{\boldsymbol{T}})_{j}=\varepsilon_{0}\left[(\nabla \cdot \boldsymbol{E}) E_{j}+(\boldsymbol{E} \cdot \nabla) E_{j}-\frac{1}{2} \nabla_{j} E^{2}\right]+\frac{1}{\mu_{0}}\left[(\nabla \cdot \boldsymbol{B}) B_{j}+(\boldsymbol{B} \cdot \nabla) B_{j}-\frac{1}{2} \nabla_{j} B^{2}\right]
        \end{equation}
        这恰恰和$\boldsymbol{f}$中的形式一致,则力密度可以用很简洁的形式表示
        \begin{equation}
            f=\nabla \cdot \stackrel{\leftrightarrow}{T}-\varepsilon_{0} \mu_{0} \frac{\partial S}{\partial t}
        \end{equation}
    \subsection{动量守恒}  
        