%!TEX program = xelatex
	
%-----------------------------导言区---------------------------%
	\documentclass[10pt,oneside,UTF8]{book}
	\usepackage[fontset=mac]{ctex}
	\usepackage[shortlabels]{enumitem}
	\usepackage{graphicx,subfigure,booktabs,multirow,caption,setspace,listings,amsmath,lineno,multicol,float,stfloats,wrapfig,amsfonts,mathrsfs}  % 直接导入常用包
	\usepackage{xcolor,colortbl,rotating,bigstrut}
	\usepackage{hyperref} % 生成书签和超链接
	\usepackage{ulem} %解决文字下划线无法自动换行的问题(暂时无用)
	\usepackage[hyperref=true,backend=biber,bibstyle=gb7714-2015,citestyle=numeric-comp,sorting=none,backref=true]{biblatex}
	\usepackage{titlesec}
		%改变section、subsection里面字体的样式。中文黑体,英文TNR。
		\newfontfamily\sectionef{Times New Roman}
		\newcommand{\sectioncf}{\CJKfamily{FZHeiTi}}
		\titleformat*{\section}{\Large\bfseries\sectioncf\sectionef}
		\titleformat*{\subsection}{\large\bfseries\sectioncf\sectionef}
	\usepackage{fancyhdr}
	\usepackage{geometry} % 用于纸张格式的设置
	\geometry{a4paper,scale=0.85,top=2cm,bottom=2.5cm,left=2.5cm,right=2.5cm}
	
	%----------------------标题与分栏线----------------------------%
	\title{\fontsize{30pt}{30pt}\textbf{电动力学笔记}}
	% 对于空格有 \! \, \: \ \quad \qquad几种形式,间隔逐渐增大
	\author
	{\kaishu 郭蒙 \\
	\kaishu 中山大学物理学院,广东,广州,510275} % \quad表示单个空格
	\date{}
	
	\setlength\columnsep{1cm} %设置分栏之后的栏间间距
	\setlength{\columnseprule}{0.5pt} % 设置分栏之后的分割线宽度
%---------------------正文--------------------------%
\begin{document}
	\maketitle  % 将前文的标题进行创建
\newpage
\pagenumbering{Roman}
\setcounter{page}{1}
\tableofcontents
\newpage
\setcounter{page}{1}
\pagenumbering{arabic}
	\chapter{电磁现象的普遍规律}
		\section{电荷和电场}
	\subsection{库伦定律}
		真空中静止点电荷Q对另一个静止点电荷Q`的作用力$\boldsymbol{F}$为
			\begin{equation}
			\boldsymbol{F}=\frac{Q Q^{\prime}}{4 \pi \varepsilon_{0} r^{3}} \boldsymbol{r}
			\end{equation}
		其中$\varepsilon_{0}$为真空介电常量。

		现代观点认为两电荷之间的作用力是通过场传递的,由上式可知电荷受到的作用力与电荷量Q成正比,我们定义电场强度$\boldsymbol{E}(x)$来表述电荷之间的作用力
		\begin{equation}
		\boldsymbol{F}=\boldsymbol{Q}^{\prime} \boldsymbol{E}
		\end{equation}
		结合两式可以直接表达出电场强度
		\begin{equation}
			\boldsymbol{E}=\frac{Q \boldsymbol{r}}{4 \pi \varepsilon_{0} r^{3}}
		\end{equation}

		在实际情况中,由于电场具有叠加性,由物体的电荷密度球积分可以得到一个几何物体周围的电场强度分布
		\begin{equation}
		\boldsymbol{E}(x)=\int_{V} \frac{\rho\left(\boldsymbol{x}^{\prime}\right) \boldsymbol{r}}{4 \pi \varepsilon_{0} r^{3}} \mathrm{~d} V^{\prime}
		\end{equation}
	\subsection{高斯定理和电场散度}
		在研究电荷和电场的关系时,我们知道电荷Q发出的电场强度的通量总是正比于Q,与附近其他电荷的存在无关,我们以点电荷周围的闭合曲面S的电场强度E的通量,由高斯定理得
		\begin{equation}
			\oint_{S} \boldsymbol{E} \cdot \mathrm{d} \boldsymbol{S}= \frac{Q}{\varepsilon_0}
		\end{equation}

		将面元投影得到闭合曲面的通量
		\begin{equation}
			\oint_{S} \boldsymbol{E} \cdot \mathrm{d} S=\frac{Q}{4 \pi \varepsilon_{0}} \oint \mathrm{d} \Omega=\frac{Q}{\varepsilon_{0}}
		\end{equation}
		其中$d \Omega$为立体角元。相应的,对于已知的空间电荷分布,对相应的电荷求积分即可得到
		\begin{equation}
			\oint_{s} E \cdot \mathrm{d} S=\frac{1}{\varepsilon_{0}} \int_{V} \rho \mathrm{d} V
		\label{fig.电场通量积分式}
		\end{equation}

		当体积V不断缩小的时候,电场的通量可以改写为散度的形式,有
		\begin{equation}
			\boxed{\boldsymbol{\nabla} \cdot \boldsymbol{E}=\frac{\rho}{\boldsymbol{\varepsilon}_{0}}}
		\end{equation}
		它是电场的一个基本微分方程,反映电场作用的局域性质:\textbf{空间某点领域上场的散度只和点上的电荷密度有关,而和其他地点的电荷分布无关}。特别的,在一般的运动电荷的前提下,远处的场不能用库伦定律表出。
	\subsection{静电场的旋度}
		旋度所反映的是场的环流性质,下面我们用库伦定律来证明电场没有旋度。在电场空间的某个环路中有下面的关系
		\begin{equation}
			\oint_{L} \boldsymbol{E} \cdot \mathrm{d} \boldsymbol{l}=\frac{Q}{4 \pi \varepsilon_{0}} \oint_{L} \frac{\mathrm{d} r}{r^{2}}=-\frac{Q}{4 \pi \varepsilon_{0}} \oint_{L} \mathrm{~d}\left(\frac{1}{r}\right)
		\end{equation}
		\begin{figure}[H]
						\centering  % width栏调节相对行宽的大小
						\includegraphics[width=0.2\linewidth]{figs/电场环路.jpg}
						\caption{电场环路} % 图片添加注释
						\label{fig.电场环路}
						\end{figure}

		观察上式得到右边的被积函数是一个全微分,绕L的回路积分值为零。因此得到
			\begin{equation}
				\boxed{\oint_{L} \boldsymbol{E} \cdot \mathrm{d} \boldsymbol{l}=0}
			\end{equation}

		将回路不断缩小,得到静电场的旋度表达
			\begin{equation}
				\boldsymbol{\nabla} \times \boldsymbol{E}=0
			\end{equation}

		电荷是电场的源,电场线从正电荷发出而终止于负电荷,在自由空间中电场线连续通过;在静电情形下电场没有旋涡状结构。
\section{电流和磁场}
		在介绍磁场性质之前,先给出两个电流分布的基本规律。
		\subsection{电荷守恒定律}
			一个系统的总电荷严格保持不变。以电流和电荷的关系用连续性方程表示为
				\begin{equation}
					\oint_{S} \boldsymbol{J} \cdot \mathrm{d} \boldsymbol{S}=\int_V \boldsymbol{\nabla} \cdot \boldsymbol{J} \mathrm{d} V=-\int_{V} \frac{\partial \rho}{\partial t} \mathrm{~d} V
				\end{equation}
			化简得到微分方程
			\begin{equation}
				\boldsymbol{\nabla} \cdot \boldsymbol{J}+\frac{\partial \rho}{\partial t}=0
			\end{equation}
			上式被称为\textbf{电流连续性方程}。特别的,当研究的对象是恒定电流时,物理量不随时间变化,此时$\partial \rho/\partial t=0$,所以在恒定电流情况下
			\begin{equation}
			\boldsymbol{\nabla} \cdot \boldsymbol{J}=0
			\end{equation}
			这一点也可以印证恒定电流的分布是无源的,流线必为闭合曲线。
		\subsection{毕奥-萨伐尔定律}
			该定律描述了两个电流之间的作用力,一个电流元在磁场中受到的力可以表示为
			\begin{equation}
			d \boldsymbol{F}=Id \boldsymbol{l}\times \boldsymbol{B}
			\end{equation}
			给出电流密度之后表示场点上的磁感应强度为
			\begin{equation}
				\boldsymbol{B}(\boldsymbol{x})=\frac{\mu_{0}}{4 \pi} \int_{V} \frac{\boldsymbol{J}\left(\boldsymbol{x}^{\prime}\right) \times \boldsymbol{r}}{r^{3}} \mathrm{~d} V^{\prime}
			\end{equation}
			当电流集中在细导线上,改写定律为
			\begin{equation}
				\boldsymbol{B}(\boldsymbol{x})=\frac{\mu_{0}}{4 \pi} \oint_{L} \frac{I \mathrm{~d} \boldsymbol{l} \times \boldsymbol{r}}{r^{3}}
			\end{equation}
	\subsection{磁场的环量和旋度}
		磁场沿闭合曲线的环量与通过闭合曲线所围曲面的电流I成正比
		\begin{equation}
		\oint_L \boldsymbol{B} \cdot d \boldsymbol{l}=\mu_0 I
		\label{fig.安培环路定律}
		\end{equation}
		上式称为安培环路定律,可以通过毕奥-萨伐尔定律证明,在此不多赘述。

		求得微分形式
		\begin{equation}
			\boldsymbol{\nabla} \times \boldsymbol{B}=\mu_{0} \boldsymbol{J}
		\end{equation}
	\subsection{磁场的散度}
		磁感应强度是无源场,对任何闭合曲面的总通量为零
		\begin{equation}
			\oint_{s} \boldsymbol{B} \cdot \mathrm{d} \boldsymbol{S}=0
		\end{equation}
		微分形式
		\begin{equation}
			\boldsymbol{\nabla} \cdot \boldsymbol{B}=0
		\end{equation}
	\subsection{磁场旋度和散度公式的证明}
\section{麦克斯韦方程组}
	在之前的章节里我们已经讨论了静电磁场的散度和旋度,下面我们考虑变化的电磁场的关系,主要是考虑新增的两个物理量之间的联系
	\begin{enumerate}
	\item 变化的磁场激发电场
	\item 变化的电场激发磁场
	\end{enumerate}

	\subsection{电磁感应定律}
		我们引入线圈L上的感应电动势,已知它和磁通量的变化率成正比。对于闭合线圈而言,感应电动势的存在表面线圈中存在电场,且感应电动势是电场强度沿闭合回路的线积分,综合上述得到
		\begin{equation}
			\oint_{L} \boldsymbol{E} \cdot d \boldsymbol{l}=\mathscr{E}= -\frac{d}{d t} \int_{S} \boldsymbol{B} \cdot d\boldsymbol{S}
		\end{equation}
		微分形式
		\begin{equation}
			\boldsymbol{\nabla} \times \boldsymbol{E}=-\frac{\partial \boldsymbol{B}}{\partial t}
		\end{equation}

		感应电场是有旋场。
	\subsection{位移电流}
		在上一节我们指出恒定电流是闭合的,在交变的情况下,电流分布不再是闭合的——可以用带电容器的电路为例,实际上该处的电流是中断的。

		恒定电流满足安培环路定律\ref{fig.安培环路定律},对式子两边取散度,由于$\nabla \cdot \nabla \times \boldsymbol{B} \equiv 0$,上式只有当$\nabla \cdot \boldsymbol{J}=0$的时候成立,而恒定电流下无磁场产生,因此两者相互矛盾。

		我们假设存在一个位移电流$\boldsymbol{J}_D$,使得它和电流$\boldsymbol{J}$合起来构成闭合的电流,满足
		\begin{equation}
		\nabla \cdot\left(\boldsymbol{J}+\boldsymbol{J}_{\mathrm{D}}\right)=0
		\label{eq.位移电流定义式}
		\end{equation}
		这样可以解决前文的矛盾,现在只需要表示出位移电流,联立
		\begin{equation}
		\begin{gathered}
			\nabla \cdot \boldsymbol{J}+\frac{\partial \rho}{\partial t}=0 \\
			\nabla \cdot \boldsymbol{E}=\frac{\rho}{\varepsilon_{0}}
		\end{gathered}
		\end{equation}
		得到$\boldsymbol{J}_D$的一个可能的表达式\footnote{实际上由式\ref{eq.位移电流定义式}可以得到多个可能的解,只是该解是最简单的求解}
		\begin{equation}
			\boldsymbol{J}_{\mathbf{D}}=\varepsilon_{0} \frac{\partial \boldsymbol{E}}{\partial t}
		\end{equation}

		\par \qquad

		综上所述,得到MaxWell方程组
		\begin{equation}
			\boxed{\begin{gathered}
		\boldsymbol{\nabla} \times \boldsymbol{E}=-\frac{\partial \boldsymbol{B}}{\partial t} \\
		\boldsymbol{\nabla} \times \boldsymbol{B}=\mu_{0} \boldsymbol{J}+\mu_{0} \varepsilon_{0} \frac{\partial \boldsymbol{E}}{\partial t} \\
		\boldsymbol{\nabla} \cdot \boldsymbol{E}=\frac{\rho}{\varepsilon_{0}} \\
		\boldsymbol{\nabla} \cdot \boldsymbol{B}=0
		\end{gathered}}
		\end{equation}

	\subsection{洛伦兹力公式}
		场对电荷体系的作用分别表述为电场和磁场之间的作用,通过库仑定律和安培定律体现,分别是
			\begin{equation}
				\begin{aligned}
				\vec{F}&=Q \vec{E} \\
				d \vec{F}&=I d \vec{l} \times \vec{B}=\vec{J} \times \vec{B} d V
				\end{aligned}
			\end{equation}

		两者相加并写作力密度的形式
			\begin{equation}
			\vec{f}=\rho \vec{E}+\vec{J} \times \vec{B}
			\end{equation}

		对于单个点电荷,很容易得到关系$\vec{J}=\rho \vec{v}$

		化简得
			\begin{equation}
				\vec{F}=q \vec{E}+q \vec{v} \times \vec{B}
			\end{equation}

\section{介质的电磁性质}
	介质是一个带电粒子系统,在介质被极化的时候,由正点中心和负电中心是否重合分为两类:重合的介质在
		\begin{figure}[H]
			\centering  % width栏调节相对行宽的大小
			\includegraphics[width=0.8\linewidth]{figs/电磁介质.jpg}
			\caption{电磁介质} % 图片添加注释
			\label{fig.电磁介质}
		\end{figure}

	\subsection{介质的极化}
		\begin{wrapfigure}{r}{4cm}%靠文字内容的右侧
      		\centering
      		\includegraphics[width=0.28\textwidth]{figs/介质极化示意图.jpeg}
      		\caption{介质的极化}
      		\label{fig.介质的极化}
      		\end{wrapfigure}
      		在外场的作用下,介质中的正负电子对会一致取向排列,形成有图的排列。如果正电荷在外,那么介质中便有等量的负电荷来维持介质的整体电中性。\textbf{这种由于极化而出现的电荷分布称为束缚电荷},以$\rho_p$表示束缚电荷密度,有关系
		\begin{equation}
			\int_{V} \rho_{\mathrm{P}} \mathrm{d} V=-\oint_{S} \boldsymbol{P} \cdot \mathrm{d} S
		\end{equation}
		对于体积分的表示,则可以表达为
			\begin{equation}
			\rho_p=- \nabla \cdot \mathbf{P}
			\end{equation}
		
		非均匀介质极化后整个介质内部都有束缚电荷出现,而在均匀介质内,束缚电荷字出现在自由电荷附近和介质界面处。我们需要特别说明的是介质分界面上束缚电荷的表示,由上图\ref{fig.介质的极化}可以得知——在两个介质的分界面,介质1和介质2分别都有本介质极化后的剩余电荷和另一个介质“漂移”过来的电荷,在分界面薄层出现的净余电荷就可以表示为$-\left(\boldsymbol{P}_{2}-\boldsymbol{P}_{1}\right) \cdot \mathrm{d} \boldsymbol{S}$,其中有束缚电荷面密度$\sigma_p$,则进一步表示为
		\begin{equation}
			\boldsymbol{\sigma}_{\mathrm{P}} \mathrm{d} S=-\left(\boldsymbol{P}_{2}-\boldsymbol{P}_{1}\right) \cdot \mathrm{d} \boldsymbol{S}
		\end{equation}
		\begin{equation}
			\sigma_{\mathrm{P}}=-\boldsymbol{e}_{\mathbf{n}} \cdot\left(\boldsymbol{P}_{2}-\boldsymbol{P}_{1}\right)
		\end{equation}
		介质内的电现象包括两个方面.一方面电场使介质极化而产生束缚电荷分布,另一方面这些束缚电荷又反过来激发电场,两者是互相制约的。\textbf{介质对宏观电场的作用就是通过束缚电荷激发电场。}

		所以整体仍然满足Maxwell方程
			\begin{equation}
				\varepsilon_{0} \nabla \cdot \boldsymbol{E}=\rho_{\mathrm{f}}+\rho_{\mathrm{p}}
				\label{eq.1.32}
			\end{equation}
		由于束缚电荷不易测量,更容易得到的是自由电荷,我们引入辅助量电位移矢量$\mathbf{D}$,定义为
		\begin{equation}
			\boldsymbol{D}=\varepsilon_{0} \boldsymbol{E}+\boldsymbol{P}
		\end{equation}
		可以看出,$\mathbf{D}$并不能代表介质中的场强,仅仅有着辅助的作用
		代入上式得到关系
		\begin{equation}
			\nabla \cdot  \boldsymbol{D}=\rho_{f}
		\end{equation}

		极化强度矢量毫无疑问与外电场强度呈某种比例关系,对于一般的各向同性线性介质,极化强度和电场强度有着线性关系
		\begin{equation}
			\boldsymbol{P}=\chi_{\mathrm{e}} \varepsilon_{0} \boldsymbol{E}
		\end{equation}
		联立上式很容易得到关系
		\begin{equation}
			\boldsymbol{D}=\varepsilon \bar{E}
		\end{equation}
		\begin{equation}
			\varepsilon=\varepsilon_{\mathrm{r}} \varepsilon_{0}, \quad \varepsilon_{\mathrm{r}}=1+\chi_{\mathrm{e}}
		\end{equation}
		$\varepsilon$ 称为介质的电容率, $\varepsilon_{\mathrm{r}}$为相对电容率。

		我们需要明确的是
		\begin{enumerate}
		\item 电场E是电场的基本物理量,代表介质中的总宏观电场强度
		\item 电位移矢量仅仅只是一个辅助量,用于方便的建立与介质自由电荷之间的关系
		\end{enumerate}

		
	\subsection{介质的磁化}
		\begin{wrapfigure}{r}{4.5cm}%靠文字内容的右侧
	      \centering
	      \includegraphics[width=0.2\textwidth]{figs/分子电流磁化.jpg}
	      \caption{分子电流磁化}
	      \end{wrapfigure}
		目前介质磁化的理论解释有两种——分子电流观点和磁荷观点,我们在这里以分子电流为例。

		为使得介绍的流程更加明晰,我们先直接写出麦克斯韦方程组在介质中的满足的方程
		\begin{equation}
		\label{eq.4.16}
			\frac{1}{\mu_{0}} \boldsymbol{\nabla} \times \boldsymbol{B}=\boldsymbol{J}_{\mathrm{f}}+\boldsymbol{J}_{\mathrm{M}}+\boldsymbol{J}_{\mathrm{P}}+\varepsilon_{0} \frac{\partial \boldsymbol{E}}{\partial t}
		\end{equation}

		其中$\boldsymbol{J}_{\mathrm{f}}$为介质中的自由电流密度,$\boldsymbol{J}_{\mathrm{M}}$为介质中的磁化电流密度,$\boldsymbol{J}_{\mathrm{P}}$为介质中的极化电流密度,后两者一并称为\textbf{诱导电流密度}。下面介绍它们的形成机理。

		对于磁化电流密度。为描述磁介质的磁化状态,引入磁化强度矢量,定义为单位体积内分子磁矩的矢量和,我们把宏观体积元内的分子看作电流环,那么只要刚刚与面边缘相交的环对介质磁化有着共贡献,那么可以得到关系
		\begin{equation}
			\oint_{L} M \cdot \mathrm{d} l=\sum_{(L \text { 内) }} I^{\prime}
		\end{equation}
		相应的微分形式为
		\begin{equation}
			J_{M}=\nabla \times M
		\end{equation}

		而对于极化电流密度而言,是电场变化导致的介质极化强度变化而产生的电流,满足
		\begin{equation}
			\frac{\partial \boldsymbol{P}}{\partial t}=\frac{\sum_{i} e_{i} \boldsymbol{v}_{i}}{\Delta V}=\boldsymbol{J}_{\mathrm{P}}
		\end{equation}

		和介质的电场规律一样,自由电流的分布可以直接通过实验测得,我们希望引入一个物理量使得它仅由介质性质和自由电流的分布决定,由上式变化得
		\begin{equation}
			\boldsymbol{\nabla} \times\left(\frac{\boldsymbol{B}}{\mu_{0}}-\boldsymbol{M}\right)=\boldsymbol{J}_{\mathrm{f}}+\frac{\partial \boldsymbol{D}}{\partial t}
		\end{equation}

		引入磁场强度$\mathbf{H}$,定义为
		\begin{equation}
			\mathbf{H}=\frac{\boldsymbol{B}}{\mu_{0}}-\mathbf{M}
		\end{equation}

		这样就有关系
		\begin{equation}
		\label{eq.4.19}
			\boldsymbol{\nabla} \times \mathbf{H}=\boldsymbol{J}_{\mathrm{f}}+\frac{\partial \mathbf{D}}{\partial t}
		\end{equation}
		从式\ref{eq.4.16}和式\ref{eq.4.19}可以看出,磁感应强度B描述了介质内整体的宏观磁场,而H仅仅只是一个辅助物理量,为能完备的描述H和B之间的关系,我们还需要介质的磁化率$\chi_M$。对于各向同性非铁磁物质而言,上式的几个物理量之间有着线性关系\footnote{为什么讨论磁介质的时候要考虑极化带来的影响,而讨论极化的时候没有与磁化相关的项}
		\begin{equation}
			\mathbf{M}=\chi_{\mathrm{M}} \mathbf{H}
		\end{equation}
		\begin{equation}
			B=\mu \mathbf{H}
		\end{equation}
		\begin{equation}
			\mu=\mu_{\mathrm{r}} \mu_{0}, \quad \mu_{\mathrm{r}}=1+\chi_{\mathrm{M}}
		\end{equation}
	\subsection{介质中的Maxwell方程组}
		\begin{equation}
		\boxed{
			\begin{aligned}
			&\nabla \times \boldsymbol{E}=-\frac{\partial \boldsymbol{B}}{\partial t} \\
			&\nabla \times \boldsymbol{H}=\boldsymbol{J}+\frac{\partial \mathbf{D}}{\partial t} \\
			&\nabla \cdot \boldsymbol{D}=\rho \\
			&\nabla \cdot \boldsymbol{B}=0
			\end{aligned}}
		\end{equation}
		式中出现的$\rho$等符号都是直接代指自由电荷和自由电流的分布。

		上述方程组想要求解还需要下面的一些关系
		\begin{equation}
			\begin{aligned}
			&\boldsymbol{D}=\varepsilon \bar{E} \\
			&\boldsymbol{B}=\mu \boldsymbol{H} \\
			&\boldsymbol{J}=\sigma \bar{E}
			\end{aligned}
		\end{equation}
		注意,这只在部分的各向同性介质中成立,对于更复杂的异性介质,可能需要张量式来描述
		\vspace*{2em}

		在介质中电场磁场相应的对应关系为
		
		\begin{figure}[H]
						\centering  % width栏调节相对行宽的大小
						\includegraphics[width=0.9\linewidth]{figs/介质中电场磁场对应关系.jpeg}
						\caption{介质中电场磁场对应关系} % 图片添加注释
						\label{fig.介质中电场磁场对应关系}
						\end{figure}
	\section{电磁场边值关系}
		我们在本节的开始直接给出积分形式的介质内麦克斯韦方程组,根据下式我们可以很直观的推导出电磁场的边值关系
			\begin{equation}
				\label{eq.1_53}
				\begin{aligned}
				&\oint_{L} \boldsymbol{E} \cdot \mathrm{d} \boldsymbol{l}=-\frac{\mathrm{d}}{\mathrm{d} t} \int_{S} \boldsymbol{B} \cdot \mathrm{d} \boldsymbol{S} \\
				&\oint_{L} \boldsymbol{H} \cdot \mathrm{d} \boldsymbol{l}=I_{\mathrm{f}}+\frac{\mathrm{d}}{\mathrm{d} t} \int_{S} \boldsymbol{D} \cdot \mathrm{d} \boldsymbol{S} \\
				&\oint_{S} \boldsymbol{D} \cdot \mathrm{d} \boldsymbol{S}=\boldsymbol{Q}_{\mathrm{f}} \\
				&\oint_{S} \boldsymbol{B} \cdot \mathrm{d} \boldsymbol{S}=0
				\end{aligned}
				\end{equation}
		\subsection{法向分量的关系}
			总电场的Maxwell方程组有
				\begin{equation}
					\varepsilon_{0} \oint_{S} \boldsymbol{E} \cdot \mathrm{d} \boldsymbol{S}=Q_{\mathrm{f}}+Q_{\mathrm{P}}
				\end{equation}
			如果在边界面上取一个厚度几乎为零的圆柱体,那么上式的面积分就等价于$\left(E_{2 \mathrm{n}}-E_{1 \mathrm{n}}\right) \Delta S$,对于上式而言,及方程左边化为面密度的物理量,即
				\begin{equation}
					\varepsilon_{0}\left(E_{2 \mathrm{n}}-E_{1 \mathrm{n}}\right)=\sigma_{\mathrm{f}}+\sigma_{\mathrm{P}}
				\end{equation}
			本式在前文讨论介质极化的时候已经提到过,见式\ref{eq.1.32}

			同样的,根据介质极化的相关结论,我们可以得到
			\begin{equation}
				P_{2 \mathrm{n}}-P_{1 \mathrm{n}}=-\sigma_{\mathrm{P}}
			\end{equation}
			\begin{equation}
				\boxed{D_{2 \mathrm{n}}-D_{1 \mathrm{n}}=\sigma_{\mathrm{f}}}
			\end{equation}

			可以看出,与介质极化相关的这三个物理量,以及三者在法向上的跃变,分别受不同的电荷面密度影响。

			相应的,对于磁场满足
			\begin{equation}
				\oint_{S} \boldsymbol{B} \cdot \mathrm{d} \boldsymbol{S}=0
			\end{equation}
			同样的利用取薄圆柱体的方法,容易得到
			\begin{equation}
				\boxed{B_{2 \mathrm{n}}=B_{1 \mathrm{n}}}
			\end{equation}
		\subsection{切向分量的关系}
		\begin{wrapfigure}{r}{4.5cm}%靠文字内容的右侧
	      \centering
	      \includegraphics[width=0.12\textwidth]{figs/磁化面电流示意图.jpg}
	      \caption{磁化面电流示意图}
	      \end{wrapfigure}
			先明确面电流分布——及体电流分布在宏观上等效于表面电流分布的一种考虑情况,我们假设面电流线密度为$\alpha$,则垂直流过一段距离的电流表示为
			\begin{equation}
				\Delta I=\alpha \Delta l
			\end{equation}

			根据式\ref{eq.1_53}第二式可以得知求解磁场的边值关系需要的两个物理量——$I_f$和$ \int_{S} \boldsymbol{D} \cdot \mathrm{d} \boldsymbol{S}$。

			我们选取介质表面一个宽度趋近于零的矩形,面电流的方向垂直于平面方向,如图\ref{fig.切向分量边值关系示意图}。
			\begin{figure}[H]
				% centering使图片居中
				\centering  % width栏调节相对行宽的大小
				\includegraphics[width=0.7\linewidth]{figs/切向分量边值关系示意图.jpg}
				\caption{切向分量边值关系示意图} % 图片添加注释
				\label{fig.切向分量边值关系示意图}
				\end{figure}
			由于这个回路的面积趋近于零且电位移矢量D有限大,则
			\begin{equation}
			\frac{\mathrm{d}}{\mathrm{d} t} \int_{S} \boldsymbol{D} \cdot \mathrm{d} \boldsymbol{S} \to 0
			\end{equation}
			另一项的推导也非常的明显
			\begin{equation}
				\oint_L \boldsymbol{H} \cdot d \boldsymbol{l} = (H_{2t}-H_{1t}) \Delta l = \alpha_f \Delta l
			\end{equation}
			即
			\begin{equation}
				H_{2 \mathrm{t}}-H_{1 \mathrm{t}}=\alpha_{f}
			\end{equation}

			刚刚给出的分量形式的边值关系,对于一般的矢量
			\begin{equation}
				\boxed{\boldsymbol{e}_{\mathrm{n}} \times\left(\boldsymbol{H}_{2}-\boldsymbol{H}_{1}\right)=\boldsymbol{\alpha}_{f}}
			\end{equation}
			同理,电场的情况就要简单很多
			\begin{equation}
				\boxed{\boldsymbol{e}_{\mathrm{n}} \times (\boldsymbol{E}_2 - \boldsymbol{E}_1) = 0}
			\end{equation}
			
			\vspace*{3em}
			对于自由电荷面密度$\sigma$和自由电流线密度${\boldsymbol{\alpha}_f}$我们能够总结出边值关系为(其中$\boldsymbol{e}_{\mathrm{n}}$是由介质1指向介质2的单位法向量。
			\begin{equation}
				\begin{aligned}
				&\boldsymbol{e}_{\mathrm{n}} \times\left(\boldsymbol{E}_{2}-\boldsymbol{E}_{1}\right)=0 \\
				&\boldsymbol{e}_{\mathrm{n}} \times\left(\boldsymbol{H}_{2}-\boldsymbol{H}_{1}\right)=\boldsymbol{\alpha} \\
				&\boldsymbol{e}_{\mathrm{n}} \cdot\left(\boldsymbol{D}_{2}-\boldsymbol{D}_{1}\right)=\sigma \\
				&\boldsymbol{e}_{\mathrm{n}} \cdot\left(\boldsymbol{B}_{2}-\boldsymbol{B}_{1}\right)=0
				\end{aligned}
				\end{equation}

	\chapter{静电场}

	\chapter{静磁场}
	
	\chapter{电磁波的传播}
		\section{平面电磁波}
    \subsection{电磁场波动方程的导出}
        考虑在没有自由电荷分布的自由空间或均匀绝缘介质中的电磁场运动形式的情况下(即$\rho=0,J=0$的情况下),麦克斯韦方程可以写作
        \begin{equation}
            \label{eq.4_1}
            \begin{gathered}
            \nabla \times \boldsymbol{E}=-\frac{\partial \mathbf{B}}{\partial t} \\
            \nabla \times \mathbf{H}=\frac{\partial \mathbf{D}}{\partial t} \\
            \nabla \cdot \boldsymbol{D}=\rho \\
            \nabla \cdot \mathbf{B}=0
            \end{gathered}
        \end{equation}

        先讨论在真空情形中$\boldsymbol{D}=\varepsilon_{0} \boldsymbol{E}, \boldsymbol{B}=\mu_{0} \mathbf{H}$的情况,联立上述两式,并结合式\ref{eq.4_1}可以得到
            \begin{equation}
                \label{eq.4_2}
                \mu_0 \frac{\partial \boldsymbol{H}}{\partial t} = \frac{\partial \boldsymbol{B}}{\partial t} = - \nabla \times \boldsymbol{E}
            \end{equation}
        上式两边同时取旋度得到
            \begin{equation}
                \nabla \times (\mu_0 \frac{\partial \boldsymbol{H}}{\partial t}) = \mu_0 \frac{ \partial }{\partial t}(\nabla \times \boldsymbol{H}) = \nabla \times (\nabla \times \boldsymbol{E})
            \end{equation}
            \begin{equation}
                -\mu_0 \varepsilon_0 \frac{\partial^2 \boldsymbol{E}}{\partial^2 t} = \nabla \times (\nabla \times \boldsymbol{E}) = \nabla(\nabla \cdot \boldsymbol{E})-\nabla^2 \boldsymbol{E}
            \end{equation}
        根据式\ref{eq.4_1}第三项和第四项可以得到$\nabla \cdot \boldsymbol{E}=0$,整理上式得
            \begin{equation}
                \label{eq.4_5}
                \nabla^2 \boldsymbol{E} - \mu_0 \varepsilon_0 \frac{\partial^2 \boldsymbol{E}}{\partial^2 t} = \nabla^2 \boldsymbol{E} - \frac{1}{c^2} \frac{\partial^2 \boldsymbol{E}}{\partial^2 t}= 0
            \end{equation}
        同理,如果对磁场进行相应的操作可以得到
        \begin{equation}
            \label{eq.4_6}
            \nabla^2 \boldsymbol{H} - \frac{1}{c^2} \frac{\partial^2 \boldsymbol{H}}{\partial^2 t}= 0
        \end{equation}

        式\ref{eq.4_5}和\ref{eq.4_6}统称为波动方程,c是电磁波在真空中的传播速度。请务必明确上面是在没有自由电荷分布的自由空间或均匀绝缘介质中才成立。\footnote{对于介质中的传播形式,我们在这里不赘述,详细请翻阅教材}
    \subsection{时谐电磁波}
        时谐电磁波,即电磁波的激发源以一定频率正弦振荡激发的电磁波称为时谐电磁波,亦称为单色波。一个非时谐的电磁波也可以使用傅立叶分析转化为时谐电磁波的叠加。

        我们讨论一定频率的电磁波,设角频率为$\omega$,电磁场方程就可以写为
        \begin{equation}
            \begin{aligned}
            &\boldsymbol{E}(\boldsymbol{x}, t)=\boldsymbol{E}(\boldsymbol{x}) \mathrm{e}^{-\mathrm{i} \omega t} \\
            &\boldsymbol{B}(\boldsymbol{x}, t)=\boldsymbol{B}(\boldsymbol{x}) \mathrm{e}^{-\mathrm{i} \omega t}
            \end{aligned}
        \end{equation}

        代入麦克斯韦方程组可以直接消去共同因子$e^{i \omega t}$得到\footnote{在$\omega \neq 0$的情况下,这四个方程不是独立的,有第一式可以推导第二式,第三式可以推导第四式}
        \begin{equation}
            \begin{gathered}
            \boldsymbol{\nabla} \times \boldsymbol{E}=\mathrm{i} \omega \mu \boldsymbol{H} \\
            \boldsymbol{\nabla} \times \boldsymbol{H}=-\mathrm{i} \omega \varepsilon \boldsymbol{E} \\
            \boldsymbol{\nabla} \cdot \boldsymbol{E}=0 \\
            \boldsymbol{\nabla} \cdot \mathbf{H}=0
            \end{gathered}
            \end{equation}
        
        上一节推导真空电磁场波动方程的流程依然适用,
        \begin{equation}
            \boxed{\nabla \times (\nabla \times \boldsymbol{E}) = \nabla(\nabla \cdot \boldsymbol{E})-\nabla^2 \boldsymbol{E}=-\nabla^2 \boldsymbol{E} = \omega^2 \mu \varepsilon \boldsymbol{E}}
        \end{equation}
        
        得到\footnote{在假定了条件$\nabla \cdot \boldsymbol{E}$之后才能够直接写成这种形式},该式称为Helmholtz方程,是一定频率电磁波的基本方程,其解$\boldsymbol{E}(x)$是电磁波场强在空间中的分布情况,每一种可能的形式称为一种波模。
        \begin{equation}
            \begin{gathered}
            \nabla^{2} \boldsymbol{E}+k^{2} \boldsymbol{E}=0 \\
            k=\omega \sqrt{\mu \varepsilon}
            \end{gathered}
            \end{equation} 
        代入麦氏方程可以得到磁场\footnote{如果用磁场的方式给出这个方程,则为\[\begin{aligned}
            \nabla^{2} \boldsymbol{B}+k^{2} \boldsymbol{B} &=0 \\
            \nabla \cdot \boldsymbol{B} &=0 \\
            \boldsymbol{E}=\frac{\mathrm{i}}{\omega \mu \varepsilon} \nabla \times \boldsymbol{B} &=\frac{\mathrm{i}}{k \sqrt{\mu \varepsilon}} \nabla \times \boldsymbol{B}
            \end{aligned}\]
            为保证文本的连贯性,不在正文中直接给出}
        \begin{equation}
            \boldsymbol{B}=-\frac{\mathrm{i}}{\omega} \nabla \times \boldsymbol{E}=-\frac{\mathrm{i}}{k} \sqrt{\mu \varepsilon} \nabla \times \boldsymbol{E}
        \end{equation}
    \subsection{平面电磁波}      
        我们假设一种最基本的解,它是平面波。设电磁波沿x轴方向传播,其场强在与x轴正交的平面上各点具有相同的值。这种情况下Helmholtz方程退化为
        \begin{equation}
            \frac{\mathrm{d}^{2}}{\mathrm{~d} x^{2}} \boldsymbol{E}(\boldsymbol{x})+k^{2} \boldsymbol{E}(\boldsymbol{x})=0
        \end{equation}
        求解上式并加上时谐项可以得到
        \begin{equation}
            \boldsymbol{E}(\boldsymbol{x}, t)=\boldsymbol{E}_{0} \mathrm{e}^{\mathrm{i}(k x-\omega t)} = \boldsymbol{E}(\cos(kx-\omega t)+i \sin(kx-\omega t))
        \end{equation}

        对于实际存在的场强应该理解为上式只取实数部分\footnote{对于此处笔者也带有一些疑问,在之后的内容里面我们可以知道复数项对于计算电磁场的瞬时能量是有影响的,在这里可以暂时将它理解为作用电磁场旋转的一个因子,在我们直接讨论场强的时候是不需要考虑的}
        \begin{equation}
            \boldsymbol{E}(\boldsymbol{x}, t)= \boldsymbol{E} \cos(kx-\omega t)
        \end{equation}

        我们直接考虑相位因子$\cos (kx - \omega t)$的意义,它给出了电磁场随位置和时间变化的关系;考虑t=0的情况时,有波峰在x=0处,在$t=1/\omega$时,这一波峰移到$x=1/k$,我们可以计算得到相速度
        \begin{equation}
            v=\frac{\omega}{k}=\frac{1}{\sqrt{\mu \varepsilon}}
        \end{equation}    
        

        
	\chapter{电磁波的辐射}
		\section{电磁场的矢势和标势}
    \subsection{用势描述电磁场}
        我们对电磁场也引入矢势和标势,我们从之前引入的磁场矢势入手\[\nabla \times \boldsymbol{B} \]在一般的情况下,磁场依然保持无源场的性质,但是在电磁波中电场同时具有有源和有旋场的性质,这意味着电场关系式中也会出现磁矢势
            \begin{equation}
                \nabla \times (\boldsymbol{E} +\frac{\partial \boldsymbol{A}}{\partial t})=0
            \end{equation}
        上面的式子可以用一个标势的负梯度描述:
            \begin{equation}
                \boldsymbol{E} + \frac{\partial \boldsymbol{A}}{\partial t} = -\nabla \varphi
            \end{equation}
        因而得到电场表达式\footnote{这一项是满足规范变换的,即\textbf{势做规范变换时,所有物理量和物理规律都应该保持不变,这种不变性称为规范不变性。}}
            \begin{equation}
                \boldsymbol{E} = -\frac{\partial \boldsymbol{A}}{\partial t} -\nabla \varphi
            \end{equation}
    \subsection{达朗贝尔方程}
        我们用麦克斯韦方程和上面的电磁场势进行推导,首先有关系
        \begin{equation*}
            \begin{gathered}
                \frac{\partial \boldsymbol{D}}{\partial t} = \varepsilon ( \frac{\partial \boldsymbol{E}}{\partial t}) = \varepsilon  \frac{\partial }{\partial t}(-\nabla \varphi -  \frac{\partial \boldsymbol{A}}{\partial t}) \\
                \nabla \times \boldsymbol{H} =  \frac{\partial \boldsymbol{D}}{\partial t} + \boldsymbol{J}
            \end{gathered}
        \end{equation*}
        可以得到
        \begin{equation}
            \nabla \times(\nabla \times \boldsymbol{A})= \mu_{0} \boldsymbol{J}-\mu_{0} \varepsilon_{0} \frac{\partial}{\partial t} \nabla \varphi-\mu_{0} \varepsilon_{0} \frac{\partial^{2} \boldsymbol{A}}{\partial t^{2}}-\nabla^{2} \varphi-\frac{\partial}{\partial t} \nabla \cdot \boldsymbol{A}=\frac{\rho}{\varepsilon_{0}}
        \end{equation}
        将真空光速的定义式代入进行化简可以得到
        \begin{equation}
            \begin{gathered}
            \nabla^{2} \boldsymbol{A}-\frac{1}{c^{2}} \frac{\partial^{2} \boldsymbol{A}}{\partial t^{2}}-\nabla\left(\nabla \cdot \boldsymbol{A}+\frac{1}{c^{2}} \frac{\partial \varphi}{\partial t}\right)=-\mu_{0} \boldsymbol{J} \\
            \nabla^{2} \varphi+\frac{\partial}{\partial t} \nabla \cdot \boldsymbol{A}=-\frac{\rho}{\varepsilon_{n}}
            \end{gathered}
        \end{equation}
        为保证电磁场的矢势和标势的形式是对称的,在这里不采用库伦规范,我们采用洛伦兹规范:
        \begin{equation}
            \nabla \cdot \boldsymbol{A} + \frac{1}{c^2} \frac{\partial \varphi}{\partial t}=0
        \end{equation}
        则达朗贝尔方程即为
        \begin{equation}
            \label{eq.5_7}
            \boxed{\begin{gathered}
            \nabla^{2} \boldsymbol{A}-\frac{1}{c^{2}} \frac{\partial^{2} \boldsymbol{A}}{\partial t^{2}}=-\mu_{0} J \\
            \nabla^{2} \varphi-\frac{1}{c^{2}} \frac{\partial^{2} \varphi}{\partial t^{2}}=-\frac{\rho}{\varepsilon_{0}} 
            \end{gathered}}
        \end{equation}
        
        方程在形式上可以看出,电荷产生标势波动,电流产生矢势波动。离开电荷电流分布区域之后,两者都以波动的形式在空间中传播。
\section{推迟势}
    由达朗贝尔导出,对于静态情形\footnote{这一节采用了格里菲斯教材的思路,回避了对达朗贝尔方程求解的具体过程,求解过程部分详细可参见郭硕鸿先生教材}
    \begin{equation}
        \begin{gathered}
            \nabla^2 \boldsymbol{\varphi} = -\frac{\rho}{\varepsilon_0} \\
            \nabla^2 \boldsymbol{\boldsymbol{A}} = -\mu_0 \boldsymbol{J}
        \end{gathered}
    \end{equation}
    有两个熟知的解
    \begin{equation}
        \begin{gathered}
            V(\boldsymbol{r}) = \frac{1}{4 \pi \varepsilon_0} \int \frac{\rho(\boldsymbol{r}^\prime)}{r}dV \\
            V(\boldsymbol{r}) = \frac{\mu_0}{4 \pi} \int \frac{\rho(\boldsymbol{J}^\prime)}{r}dV
        \end{gathered}
    \end{equation}
    其中r为源点到场点的距离。而在非静止的情况下,由于电磁波益光速传播,当一个事件从源点传到场点的时候,中间应该经过了一段推迟的时间$r/c$,或者我们可以理解成在t时刻场点受到的作用实际上来自$t-r/c$时刻的源点,我们将它称为推迟势。

    我们直接由静态情况下的解猜测推迟势的解为
    \begin{equation}
        \boxed{V(\boldsymbol{r}, t)=\frac{1}{4 \pi \varepsilon_{0}} \int \frac{\rho\left(\boldsymbol{r}^{\prime}, t_{\mathrm{r}}\right)}{r} \mathrm{~d} \tau^{\prime}, \quad \boldsymbol{A}(\boldsymbol{r}, t)=\frac{\mu_{0}}{4 \pi} \int \frac{\boldsymbol{J}\left(\boldsymbol{r}^{\prime}, t_{r}\right)}{r} \mathrm{~d} \tau^{\prime}}
    \end{equation}
    其中$t_r=t-r/c$
    
    下面我们来验证这一假设解确实是达朗贝尔方程的解\footnote{笔者更喜欢把大段的数学推导放在脚注以保持物理图像的连贯性。在计算 $V(\boldsymbol{r}, t)$ 的拉普拉斯算子时, 特别要注意的是积分在两处依赖 $r$ : 显含在分母中的 $\left(r=\left|\boldsymbol{r}-\boldsymbol{r}^{\prime}\right|\right)$ 和隐含在分子中的 $t_{\mathrm{r}} \equiv t-r / c$$$\nabla V=\frac{1}{4 \pi \varepsilon_{0}} \int\left[(\nabla \rho) \frac{1}{2}+\rho \nabla\left(\frac{1}{r}\right)\right] \mathrm{d} \tau^{\prime}$$和$$\nabla \rho=\dot{\rho} \nabla t_{r}=-\frac{1}{c} \dot{\rho} \nabla r$$(式中的点表示对时间的微分)现在有 $\nabla r=\hat{r}$ 和 $\nabla(1 / r)=-\hat{r} / r^{2}$ , 故$$\nabla V=\frac{1}{4 \pi \varepsilon_{0}} \int\left[-\frac{\dot{\rho}}{c} \frac{\hat{r}}{r}-\rho \frac{\hat{r}}{r}\right] \mathrm{d} \tau^{\prime}$$取散度,$$\nabla^{2} V=\frac{1}{4 \pi \varepsilon_{0}} \int\left\{-\frac{1}{c}\left[\frac{\hat{r}}{r} \cdot(\nabla \dot{\rho})+\dot{\rho} \nabla \cdot\left(\frac{\hat{r}}{r}\right)\right]-\left[\frac{\hat{r}}{r^{2}} \cdot(\nabla \rho)+\rho \nabla \cdot\left(\frac{\hat{r}}{r^{2}}\right)\right]\right\} \mathrm{d} \tau^{\prime}$$但$$\nabla \dot{\rho}=-\frac{1}{c} \ddot{\rho} \nabla r=-\frac{1}{c} \ddot{\rho} \hat{r}$$这同式 (10.21) 中的一样, 并且$$\nabla \cdot\left(\frac{\hat{r}}{r}\right)=\frac{1}{r^{2}}$$和$$\nabla \cdot\left(\frac{\hat{r}}{r^{2}}\right)=4 \pi \delta^{3}(r)$$故有\[\nabla^{2} V=\frac{1}{4 \pi \varepsilon_{0}} \int\left[\frac{1}{c^{2}} \frac{\ddot{\rho}}{2}-4 \pi \rho \delta^{3}(\boldsymbol{r})\right] \mathrm{d} \tau^{\prime}=\frac{1}{c^{2}} \frac{\partial^{2} V}{\partial t^{2}}-\frac{1}{\varepsilon_{0}} \rho(\boldsymbol{r}, t)\]证明了推迟势确实是达朗贝尔方程的解}

    根据推迟势公式,当$\rho$和$\boldsymbol{J}$给定之后就可以得到势,而根据下式可以算出电磁场
    \begin{equation}
        \begin{gathered}
            \boldsymbol{B} = \nabla \times \boldsymbol{A} \\
            \boldsymbol{E} = - \frac{\partial \boldsymbol{A}}{\partial t} - \nabla \varphi
        \end{gathered}
    \end{equation}
    事实上场会反作用于电荷电流分布,故激发区的电荷电流不能随意分布,我们之后为详细介绍这一点。
\section{几种常见的辐射形式}
    \footnote{本段内容来自格里菲斯版电动力学内容}我们已知球壳的面积是$4 \pi r^2$,即能流密度的减少量不能快于$1/r^2$:因为如果它以$1/r^3$的速度减少,则无限远处能流密度的积分应该是零——这是不符合常理的,因为能流密度在任意远的地方积分都应该相等。下面我们以静止的电荷为例,由库伦定律可知电场以$1/r^2$减少,由毕奥-萨法尔定律可知磁场也以$1/r^2$(或更快)的方式减少,则能流密度以$1/r^4$的速度减少,故\textbf{静止的源不产生辐射}。但与时间相关的场,包含一些以$1/r$变化的项,正是这些项导致了电磁辐射。
    \subsection{计算辐射场的一般公式}
        \textbf{计算辐射场的基本是推迟势公式}

        对于推迟势公式,以矢势为例\[\quad \boldsymbol{A}(\boldsymbol{r}, t)=\frac{\mu_{0}}{4 \pi} \int \frac{\boldsymbol{J}\left(\boldsymbol{r}^{\prime}, t_{r}\right)}{r} \mathrm{~d} \tau^{\prime}\]计算得到电磁场的矢势之后,就可以经过下式计算电磁场量
        \begin{equation}
            \boxed{\begin{gathered}
                \boldsymbol{B} = \nabla \times \boldsymbol{A} \\
                \boldsymbol{E} = \frac{i c}{k} \nabla \times \boldsymbol{B}
            \end{gathered}}
        \end{equation}

        在矢势公式中有三个距离的线度需要纳入考虑:电荷分布区域的线度$l$,波长$\lambda$,电荷到场点的距离$r$。本节中我们都研究的是小区域内电荷产生的辐射,即满足\[l \ll \lambda, \quad l \ll r\]而对于后两个量同样可以按照线度的大小关系分为:
        \begin{enumerate}[(1).]
            \item 近区 ($r \ll \lambda$),在近区内电磁场保持着恒定场的主要特点
            \item 感应区 ($r ~ \lambda$),性质介于两者之间
            \item 辐射区 ($r \gg \lambda$),在辐射区内电磁场变为横向的辐射场
        \end{enumerate}
        
        我们对矢势中的$r$展开得到\footnote{Unfinished.啊啊啊啊复习不完了复习不完了,这里跳过了展开步骤,以后有空补上}
        \begin{equation}
            \label{eq.5_13}
            \boldsymbol{A}(x) = \frac{\mu_0 e^{ikR}}{4 \pi R}\int_V \boldsymbol{J}(x^\prime)(1-ik\mathbf{e}_R \cdot x^\prime + \cdots)dV^\prime
        \end{equation}
    \subsection{电偶极辐射}
        电偶极辐射来源于\ref{eq.5_13}展开式中的第一项
        \begin{equation}
            \boldsymbol{A} = \frac{\mu_0 e^{ikR}}{4 \pi R}\int_V \boldsymbol{J}(x^\prime) dV^\prime
        \end{equation}
        其中积分项就是电偶极矩对时间的偏导数\[\int_V \boldsymbol{J}(x^\prime)dV^\prime= \sum q \boldsymbol{v}=\frac{d}{d t}\sum q \boldsymbol{x}=\frac{d \boldsymbol{p}}{d t}=\boldsymbol{\dot{p}}\]可以直接写出电偶极辐射的公式
        \begin{equation}
            \boldsymbol{A}(x) = \frac{\mu_0 e^{i k R}}{4 \pi R}\boldsymbol{\dot{p}}
        \end{equation}
        根据势的公式计算电磁场的时候,因为分母中的R是我们专门展开需要的,故$\nabla$项就只需要作用在相因子上,意味着对于以下算符可以代换为
        \begin{equation}
            \label{eq.5_17}
            \boxed{\begin{gathered}
                \nabla \to i k \boldsymbol{e}_R \\
                \frac{\partial }{\partial t} \to -i \omega 
            \end{gathered}}
        \end{equation}

        可以计算得到辐射场\footnote{我们都将结果写作只含$\varepsilon_0$和k的形式,用关系代换掉$\mu_0$}
        \begin{equation}
            \begin{aligned}
            \boldsymbol{B} &=\boldsymbol{\nabla} \times \boldsymbol{A}=\frac{\mathrm{i} \mu_{0} k}{4 \pi R} \mathrm{e}^{i k R} \boldsymbol{e}_{R} \times \dot{\boldsymbol{p}}=\frac{1}{4 \pi \varepsilon_{0} c^{3} R} \mathrm{e}^{\mathrm{i} k R} \ddot{\boldsymbol{p}} \times \boldsymbol{e}_{R} \\
            \boldsymbol{E} &=\frac{\mathrm{i} c}{k} \boldsymbol{\nabla} \times \boldsymbol{B}=c \boldsymbol{B} \times \boldsymbol{e}_{R}=\frac{\mathrm{e}^{\mathrm{i} k R}}{4 \pi \varepsilon_{0} c^{2} R}\left(\ddot{\boldsymbol{p}} \times \boldsymbol{e}_{R}\right) \times \boldsymbol{e}_{R}
            \end{aligned}
        \end{equation}
        在这种情况下,辐射区的电磁场$~1/R$,能流密度$~1/R^2$,对球面积分之后总功率与球的半径无关,保证了电磁能量可以传播到任意远处,和之前的理论吻合。
        
        \subsubsection{辐射能流、功率与角分布}
            电偶极辐射的平均能流密度可以计算得
            \begin{equation}
                \begin{aligned}
                \overline{\boldsymbol{S}} &=\frac{1}{2} \operatorname{Re}\left(\boldsymbol{E}^{*} \times \boldsymbol{H}\right)=\frac{c}{2 \mu_{0}} \operatorname{Re}\left[\left(\boldsymbol{B}^{*} \times \boldsymbol{e}_{R}\right) \times \boldsymbol{B}\right] \\
                &=\frac{c}{2 \mu_{0}}|\boldsymbol{B}|^{2} \boldsymbol{e}_{R}=\frac{|\ddot{p}|^{2}}{32 \pi^{2} \varepsilon_{0} c^{3} R^{2}} \sin ^{2} \theta \boldsymbol{e}_{R} 
                \end{aligned}
                \end{equation}
            式中$\sin^2$项即体现了辐射的角分布,可以用平均能流密度对球面积分可以得到总辐射功率\footnote{可以用整个球的立体角定义来记忆立体角的积分公式\[\frac{d S}{d\Omega}=\frac{S}{\Omega}=\frac{4 \pi R^2}{4 \pi}=R^2\]}
            \begin{equation}
                \begin{aligned}
                P &=\oint|\overline{\boldsymbol{S}}| R^{2} \mathrm{~d} \Omega \\
                &=\frac{|\ddot{\boldsymbol{p}}|^{2}}{32 \pi^{2} \varepsilon_{0} c^{3}} \oint \sin ^{2} \theta \mathrm{d} \Omega \\
                &=\frac{1}{4 \pi \varepsilon_{0}} \frac{|\ddot{\boldsymbol{p}}|^{2}}{3 c^{3}}
                \end{aligned}
            \end{equation}
            
            结合关系式\ref{eq.5_17}可以得到,总辐射功率正比于$\omega^4$,当电偶极子振荡频率增高的时候辐射功率迅速增加。
        
        \subsubsection{短天线辐射和辐射电阻}\footnote{Unfinished:留坑}
    \subsection{高频电流分布的磁偶极矩和电四极矩}
        \textcolor[RGB]{143,143,143}{笔者还没搞明白对称张量分离的办法,格里菲斯上的推导办法还没来得及看,这里先直接给出公式和结论}
        
        现在我们计算矢势展开的第二项
        \begin{equation}
            \boldsymbol{A} = \frac{- i k \mu_0 e^{ikR}}{4 \pi R}\int_V \boldsymbol{J}(x^\prime)(\mathrm{e}_R \cdot x^\prime) dV^\prime
        \end{equation}
        直接给出结果
        \begin{equation}
            \boldsymbol{A}(\boldsymbol{x})=-\frac{\mathrm{i} k \mu_{0} \mathrm{e}^{\mathrm{i} k R}}{4 \pi R}\left(-\boldsymbol{e}_{R} \times m+\frac{1}{6} \boldsymbol{e}_{R} \cdot \dot{\mathscr{D}_{ij}}\right)
        \end{equation}
        其中$\mathscr{D_{ij}}$为体系的电四极矩\[\mathscr{D}_{ij}=\sum3q\boldsymbol{x^\prime}\boldsymbol{x^\prime}\]

        式中括号第一项为磁偶极辐射势,第二项为电四极辐射势。他们在电磁矢势的同一级展开式中体现。
    \subsection{磁偶极辐射}
        我们考虑磁偶极辐射的性质,即
        \begin{equation}
            \boldsymbol{A}(x) = \frac{ik \mu_0 e^{ik R}}{4 \pi R}\boldsymbol{e}_R \times \boldsymbol{m}
        \end{equation}
        计算电磁场\footnote{这里和之前一样,都使用了\[( \frac{\partial }{\partial t})^2=(-i \omega)^2=-\omega^2\]结合$k$的定义来化简,其中负号吸收进了矢积改变了乘积方向}
        \begin{equation}
            \begin{aligned}
            \boldsymbol{B} &=\boldsymbol{\nabla} \times \boldsymbol{A}=\mathrm{i} k \boldsymbol{e}_{R} \times \boldsymbol{A} \\
            &=k^{2} \frac{\mu_{0} \mathrm{e}^{\mathrm{i} k R}}{4 \pi R}\left(\boldsymbol{e}_{R} \times \boldsymbol{m}\right) \times \boldsymbol{e}_{R} \\
            &=\frac{\mu_{0} \mathrm{e}^{\mathrm{i} k R}}{4 \pi c^{2} R}\left(\ddot{\boldsymbol{m}} \times \boldsymbol{e}_{R}\right) \times \boldsymbol{e}_{R} \\
            \boldsymbol{E} &=\frac{i c}{k} \nabla \times \boldsymbol{B} =-\frac{\mu_{0} \mathrm{e}^{\mathrm{i} k R}}{4 \pi c R}\left(\ddot{\boldsymbol{m}} \times \boldsymbol{e}_{R}\right)
            \end{aligned}
        \end{equation}

        我们知道电场和磁场在真空中有良好的对称性,我们比较上式和\ref{eq.5_17}可以发现只需以下代换就可以使两者等价
        \begin{equation}
            \begin{gathered}
                \boldsymbol{p} \to \frac{\boldsymbol{m}}{c} \\
                \boldsymbol{E} \to c \boldsymbol{B} \\
                c \boldsymbol{B} \to - \boldsymbol{E}
            \end{gathered}
        \end{equation}
        在对称性的体现上,即得在自由空间中, 麦氏方程组对变换 $\boldsymbol{E} \rightarrow c \boldsymbol{B}, c \boldsymbol{B} \rightarrow-\boldsymbol{E}$ 是对称的。若电磁场 $\boldsymbol{E}(\boldsymbol{x}, \boldsymbol{t}), \boldsymbol{B}(\boldsymbol{x}, \boldsymbol{t})$ 是麦克斯韦方程组的解,则代换后的电磁场也是麦克斯韦方程组的解。

        可以用代换直接得到磁偶极辐射的能流密度
        \begin{equation}
            \boldsymbol{S}=\frac{\mu_{0} \omega^{4}|m|^{2}}{32 \pi^{2} c^{3} R^{2}} \sin ^{2} \theta \boldsymbol{e}_{R}
        \end{equation}
        总辐射功率
        \begin{equation}
            P =\frac{\mu_0 \omega^4 |\boldsymbol{m}|^2}{12 \pi c^3}
        \end{equation}
    \subsection{电四极辐射}
        对于第二个展开项
        \begin{equation}
            \boldsymbol{A}(\boldsymbol{x})=-\frac{\mathrm{i} k \mu_{0} \mathrm{e}^{\mathrm{i} k R}}{24 \pi R}\boldsymbol{e}_{R} \cdot \dot{\mathscr{D}_{ij}}
        \end{equation}
        本节也直接给出结果,定义矢量\[\mathscr{D}_{ij}^\prime = \boldsymbol{e}_R \cdot \mathscr{D}_{ij}\]则辐射区的电磁场为
        \begin{equation}
            \begin{gathered}
                \boldsymbol{B}=\mathrm{i} k \boldsymbol{e}_{R} \times \boldsymbol{A}=\frac{\mathrm{e}^{\mathrm{i} k R}}{24 \pi \varepsilon_{0} c^{4} R} \dddot{\mathscr{D}_{ij}^\prime} \times \boldsymbol{e}_{R} \\
                \boldsymbol{E}=c \boldsymbol{B} \times \boldsymbol{e}_{R}=\frac{\mathrm{e}^{\mathrm{i} k R}}{24 \pi \varepsilon_{0} c^{3} R}\left(\dddot{\mathscr{D}_{ij}^\prime} \times \boldsymbol{e}_{R}\right) \times \boldsymbol{e}_{R}
            \end{gathered}
        \end{equation}
        辐射平均能流密度暂且不表。直接给出结论:电四极辐射和磁偶极辐射的功率是同数量级的。
\section{电磁波的衍射}
    \subsection{基尔霍夫公式}
        电磁场任一分量都满足亥姆霍兹方程,即\[\nabla^2 \psi +k^2 \psi =0\]如果我们忽略其他电磁场分量的影响,把$\psi$看作一个标量场,用边界上的$\psi$$ \frac{\partial \psi}{\partial n}$表出边界条件,这种理论就是标量衍射理论。

        可以用格林函数变形得到
        \begin{equation}
            \begin{aligned}
            \phi(\boldsymbol{x}) &=-\frac{1}{4 \pi} \oint_{S}\left[\psi\left(\boldsymbol{x}^{\prime}\right) \boldsymbol{\nabla}^{\prime} \frac{\mathrm{e}^{\mathrm{i} k r}}{r}-\frac{\mathrm{e}^{\mathrm{i} k r}}{r} \nabla^{\prime} \psi\left(\boldsymbol{x}^{\prime}\right)\right] \cdot \mathrm{d} \boldsymbol{S}^{\prime} \\
            &=-\frac{1}{4 \pi} \oint_{S} \frac{\mathrm{e}^{\mathrm{i} k r}}{r} \boldsymbol{e}_{\mathrm{n}} \cdot\left[\nabla^{\prime} \psi+\left(\mathrm{i} k-\frac{1}{r}\right) \frac{\boldsymbol{r}}{r} \psi\right] \mathrm{d} S^{\prime}
            \end{aligned}
            \end{equation}
        上式为\textbf{基尔霍夫公式},是惠更斯原理的数学表示。
    \subsection{夫琅和费衍射}
        碍于记录的时间原因,这里直接给出结果\footnote{Unfinished}
        \begin{equation}
            \begin{aligned}
            \psi(\boldsymbol{x}) &=-\frac{\mathrm{i} \psi_{0} \mathrm{e}^{\mathrm{i} k R}}{4 \pi R} \int_{S_{0}} \mathrm{e}^{\mathrm{i}\left(\boldsymbol{k}_{1}-\boldsymbol{k}_{2}\right) \cdot x^{\prime}}\left(\boldsymbol{k}_{1}+\boldsymbol{k}_{2}\right) \cdot \boldsymbol{e}_{\mathrm{n}} \mathrm{d} S^{\prime} \\
            &=-\frac{\mathrm{i} \psi_{0} \mathrm{e}^{\mathrm{i} k R}}{4 \pi R} \int_{S_{0}} \mathrm{e}^{\mathrm{i}\left(\boldsymbol{k}_{1}-\boldsymbol{k}_{2}\right) \cdot x^{\prime}}\left(\cos \theta_{1}+\cos \theta_{2}\right) \mathrm{d} S^{\prime}
            \end{aligned}
            \end{equation}
        式中 $\theta_{1}$ 为入射波矢 $\boldsymbol{k}_{1}$ 与法线方向 $\boldsymbol{e}_{\mathrm{n}}$ 的夹角, $\theta_{2}$为衍射波矢 $\boldsymbol{k}_{2}$ 与 $\boldsymbol{e}_{\mathrm{n}}$ 的夹角. $\cos \theta_{1}+\cos \theta_{2}$ 称为倾斜因子。
\section{电磁场的动量}
    \subsection{麦克斯韦应力张量}
        我们已知单位体积电荷收到的力可以表示为
        \begin{equation}
            \boldsymbol{f} = \rho \boldsymbol{E} + \boldsymbol{J} \times \boldsymbol{B}
        \end{equation}
        用麦克斯韦方程组展开并且加入一项$(\nabla \cdot \boldsymbol{B})\boldsymbol{B}$以保持结果中电场和磁场形式的对称(因为$\nabla \cdot \boldsymbol{B}$等于零,对结果没有任何影响),这样就有
        \begin{equation}
            \boldsymbol{f}=\varepsilon_{0}[(\nabla \cdot \boldsymbol{E}) \boldsymbol{E}+(\boldsymbol{E} \cdot \nabla) \boldsymbol{E}]+\frac{1}{\mu_{0}}[(\nabla \cdot \boldsymbol{B}) \boldsymbol{B}+(\boldsymbol{B} \cdot \nabla) \boldsymbol{B}]- \frac{1}{2} \nabla\left(\varepsilon_{0} E^{2}+\frac{1}{\mu_{0}} B^{2}\right)-\varepsilon_{0} \frac{\partial}{\partial t}(\boldsymbol{E} \times \boldsymbol{B})
        \end{equation}
        引入\textbf{麦克斯韦应力张量\footnote{一般二阶张量可以用双箭头表示\[\overleftrightarrow{T}\]}}可以大幅简化上式的形式\footnote{其是一个对称张量,或者直接用一个矩阵表示为\[T_{i j}=\left(\begin{array}{ccc}
            \epsilon_{0}\left(E_{x}^{2}-E^{2} / 2\right)+\frac{1}{\mu_{0}}\left(B_{x}^{2}-B^{2} / 2\right) & \epsilon_{0} E_{x} E_{y}+\frac{1}{\mu_{0}}\left(B_{x} B_{y}\right) & \epsilon_{0} E_{x} E_{z}+\frac{1}{\mu_{0}}\left(B_{x} B_{z}\right) \\
            \epsilon_{0} E_{x} E_{y}+\frac{1}{\mu_{0}}\left(B_{x} B_{y}\right) & \epsilon_{0}\left(E_{y}^{2}-E^{2} / 2\right)+\frac{1}{\mu_{0}}\left(B_{y}^{2}-B^{2} / 2\right) & \epsilon_{0} E_{y} E_{z}+\frac{1}{\mu_{0}}\left(B_{y} B_{z}\right) \\
            \epsilon_{0} E_{x} E_{z}+\frac{1}{\mu_{0}}\left(B_{x} B_{z}\right) & \epsilon_{0} E_{y} E_{z}+\frac{1}{\mu_{0}}\left(B_{y} B_{z}\right) & \epsilon_{0}\left(E_{z}^{2}-E^{2} / 2\right)+\frac{1}{\mu_{0}}\left(B_{z}^{2}-B^{2} / 2\right)
            \end{array}\right)\]张量中的ij元素诠释为,朝著i-轴方向,施加于j-轴的垂直平面,单位面积的作用力\\ \textcolor[RGB]{143,143,143}{Unfinished.这里矩阵的间距太窄不够美观,有时间修复}}:
        \begin{equation}
            T_{i j} \equiv \varepsilon_{0}\left(E_{i} E_{j}-\frac{1}{2} \delta_{i j} E^{2}\right)+\frac{1}{\mu_{0}}\left(B_{i} B_{j}-\frac{1}{2} \delta_{i j} B^{2}\right)
        \end{equation}
        麦克斯韦应力张量矢量写法和标量写法之间有关系
        \begin{equation}
            (a \cdot \overleftrightarrow{T})_{j}=\sum_{i=x, y, z} a_{i} T_{i j}
        \end{equation}
        特别的,对于$\overleftrightarrow{T}$散度的第j个分量可以表示为
        \begin{equation}
            (\nabla \cdot \stackrel{\leftrightarrow}{\boldsymbol{T}})_{j}=\varepsilon_{0}\left[(\nabla \cdot \boldsymbol{E}) E_{j}+(\boldsymbol{E} \cdot \nabla) E_{j}-\frac{1}{2} \nabla_{j} E^{2}\right]+\frac{1}{\mu_{0}}\left[(\nabla \cdot \boldsymbol{B}) B_{j}+(\boldsymbol{B} \cdot \nabla) B_{j}-\frac{1}{2} \nabla_{j} B^{2}\right]
        \end{equation}
        这恰恰和$\boldsymbol{f}$中的形式一致,则力密度可以用很简洁的形式表示
        \begin{equation}
            f=\nabla \cdot \stackrel{\leftrightarrow}{T}-\varepsilon_{0} \mu_{0} \frac{\partial S}{\partial t}
        \end{equation}
    \subsection{动量守恒}  
        
	% \section{张量代数}
	\subsection{矢量代数}
		\subsubsection*{坐标基矢的点积与矢积\label{sec.app_1_1_1}}
			我们以常见的笛卡尔坐标系(默认为右手坐标系)为例,有矢量点积的关系
			\begin{equation}
				\boldsymbol{e}_{i} \cdot \boldsymbol{e}_{j}=\sigma_{ij}= \begin{cases}0 & (i \neq j) \\ 1 & (i=j)\end{cases}
			\end{equation}
			这里我们记\textbf{二阶对称$\sigma$ 符号为}
				\begin{equation}
					\delta_{i j}= \begin{cases}0 & (i \neq j) \\ 1 & (i=j)\end{cases}
				\end{equation}
			矢量的矢积为
				\begin{equation}
					\boldsymbol{e}_{i} \times \boldsymbol{e}_{j}=\sum_{k=1}^{3} \varepsilon_{i j k} \boldsymbol{e}_{k}
				\end{equation}
			这里有三阶完全反对称符号$\varepsilon_{ijk}$的定义为
			\begin{equation}
				\left\{\begin{array}{l}
				\varepsilon_{i j k}=0 \quad(i=j \text { 或 } j=k \text { 或 } k=i) \\
				\varepsilon_{123}=\varepsilon_{231}=\varepsilon_{312}=-\varepsilon_{213}=-\varepsilon_{132}=-\varepsilon_{321}=1
				\end{array}\right.
			\end{equation}
		\subsubsection*{任意矢量的点积与矢积}
			对于三维欧式空间的任意矢量可以表示为
				\begin{equation}
					\left\{\begin{array}{l}
					\boldsymbol{a}=a_{1} \boldsymbol{e}_{1}+a_{2} \boldsymbol{e}_{2}+a_{3} \boldsymbol{e}_{3}=\sum_{i=1}^{3} a_{i} \boldsymbol{e}_{i} \\
					\boldsymbol{b}=b_{1} \boldsymbol{e}_{1}+b_{2} \boldsymbol{e}_{2}+b_{3} \boldsymbol{e}_{3}=\sum_{i=1}^{3} b_{i} \boldsymbol{e}_{i}
					\end{array}\right.
				\end{equation}
			
			根据上节\ref{sec.app_1_1_1}可以表示出
				\begin{equation}
					\boldsymbol{a} \cdot \boldsymbol{b}=\sum_{i=1}^{3} \sum_{j=1}^{3} a_{i} b_{j} \boldsymbol{e}_{i} \cdot \boldsymbol{e}_{j}=\sum_{i=1}^{3} \sum_{j=1}^{3} a_{i} b_{j} \delta_{i j}
				\end{equation}
				\begin{equation}
					\boldsymbol{a} \times \boldsymbol{b}=\sum_{i=1}^{3} \sum_{j=1}^{3} a_{i} b_{j} \boldsymbol{e}_{i} \times \boldsymbol{e}_{j}=\sum_{i=1}^{3} \sum_{j=1}^{3} \sum_{k=1}^{3} \varepsilon_{i j k} a_{i} b_{j} e_{k}
				\end{equation}

			上面的矢积使用更直白的分量式进行表示
			\begin{equation}
				(\boldsymbol{a} \times \boldsymbol{b})_{i}=\sum_{j=1}^{3} \sum_{k=1}^{3} \varepsilon_{i j k} a_{j} b_{k}
			\end{equation}
		\subsubsection*{赝矢量和赝标量}
			








  

	%%%%%----------注解区------------%%%%
	% Note=22/6/20引入143灰色作为Question注解
\end{document}