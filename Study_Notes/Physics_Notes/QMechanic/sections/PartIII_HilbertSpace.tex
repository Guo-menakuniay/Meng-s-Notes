\section{Hilbert空间}
	在N维空间中,可以简单地用对应于N个正交归一基矢的分量,可以用行列矩阵来表示
	\begin{equation}
		|\alpha\rangle \rightarrow \mathbf{a}=\left(\begin{array}{l}
		a_{1} \\
		a_{2} \\
		\vdots \\
		a_{N}
		\end{array}\right)
	\end{equation}
	
	同样可以定义两个矢量的内积
	\begin{equation}
		\langle\alpha \mid \beta\rangle=a_{1}^{*} b_{1}+a_{2}^{*} b_{2}+\cdots a_{N}^{*} b_{N}
	\end{equation}

	矩阵变换可以写做
	\begin{equation}
		|\beta\rangle=T|\alpha\rangle \rightarrow \mathbf{b}=\mathbf{T a}=\left(\begin{array}{cccc}
		t_{11} & t_{12} & \cdots & t_{1 N} \\
		t_{21} & t_{22} & \cdots & t_{2 N} \\
		\vdots & \vdots & & \vdots \\
		t_{N 1} & t_{N 2} & \cdots & t_{N N}
		\end{array}\right)\left(\begin{array}{l}
		a_{1} \\
		a_{2} \\
		\vdots \\
		a_{N}
		\end{array}\right)
	\end{equation}

	将\textbf{所有在特定区域的平方可积函数的集合}
		\begin{equation}
			f(x) \quad \text { 满足 } \quad \int_{a}^{b}|f(x)|^{2} d x<\infty
		\end{equation}
	\textbf{构成的一个矢量空间,称为希尔伯特空间。}在量子力学中,波函数是处于希尔伯特空间中的。

	定义两个函数的内积
	\begin{equation}
		\langle f \mid g\rangle \equiv \int_{a}^{b} f(x)^{*} g(x) d x
	\end{equation}

	如果一个函数与自身的内积为 1 , 我们称之为归一化的; 如果两个函数的内积为 0 , 那 么这两个函数是正交的; 如果一组函数即是归一的也是相互正交的, 称之它们为正交归一的。
	\begin{equation}
		\left\langle f_{m} \mid f_{n}\right\rangle=\delta_{m v}
	\end{equation}
\section{可观测量}
	\subsection{Hermitian算符}
		对于一个可观测量,一次测量结果的期望值和多次测量的平均值应当相同
		\begin{equation}
			\langle\psi \mid \hat{Q} \psi\rangle=\langle\hat{Q} \psi \mid \psi\rangle
		\end{equation}
		我们称这样的算符为Hermitian算符,Hermitian算符的期望值是实数,可观测量由厄密算符表示。

		举个例子,我们可以证明动量算符是厄密算符\footnote{通常会采用到分部积分法}
		\begin{equation}
			\langle f \mid \hat{p} g\rangle=\int_{-\infty}^{\infty} f^{*} \frac{h}{i} \frac{d g}{d x} d x=\left.\frac{h}{i} f^{*} g\right|_{-\infty} ^{\infty}+\int_{-\infty}^{\infty}\left(\frac{h}{i} \frac{d f}{d x}\right)^{*} g d x=\langle\hat{p} f \mid g\rangle
		\end{equation}
	\subsection{定值态}
		通常的, 当你对全同体系组成的系综测量一个可观测量 $Q$, 每个体系都处于相同的状态 $\Psi$, 每 次测量并不能得到相同的结果——这就是量子力学中的不确定性问题: 是否能够制备一个态 使得每一次观测 $Q$ 都一定得到同样的值?如果你喜欢, 可以称这样的态为可观测量 $Q$ 的定值态。\footnote{实际上, 我们已经知道一个例子:哈密顿的定态是定值态:测量一个粒子处于定态 $\Psi_{n}$ 时的总能量, 必定得到相应的 “允许的” 能量 $E_{n}$}

		则对于
		\begin{equation}
			\hat{Q} \Psi=q \Psi
		\end{equation}
		称为算符$\hat{Q}$的本征值方程,q是对应的本征值。有时候两个(或者多个)线性独立的本征函数具有相同的本征值,这种情况称为谱的简并。

	\subsection{Hermitian算符的本征函数}
		对于厄密算符的本征函数(也即可观测的定值态)分成两类情况
			\begin{enumerate}
			\item 如果谱是连续的,那么本征函数是不可归一化的,并且不能代表可能的波函数 
			\item 如果谱是分立的,那么它们的本征值是实数且不同本征值的本征函数是正交的
			\end{enumerate}

		某些算符只有分立谱,某些算符只有连续谱,某些算符两者都具有。
			\subsubsection{分立谱}
				数学上厄密算符归一化的本征函数有两个性质,我们在前文也已经提到,这里在给出性质的同时给出证明
					\paragraph*{定理1:它们的本征值是实数\\}
						\par
						证明: 假设
						\begin{equation}
						\hat{Q} f=q f
						\end{equation}
						\begin{equation}
						\langle f \mid \hat{Q} f\rangle=\langle\hat{Q} f \mid f\rangle
						\end{equation}
						( $\hat{Q}$ 是厄密算符)。那么有
						\begin{equation}
						q\langle f \mid f\rangle=q^{*}\langle f \mid f\rangle
						\end{equation}
						( $q$ 是一个数, 所以它可以移出积分号外, 并且因为内积的左侧是右侧函 数的复共轭 (等式 3.6) 所以在右边 $q$ 也同样移出)但是 $\langle f \mid f\rangle$ 不能是 0 ( $f(x)=0$ 不是正当的本征函数), 所以 $q=q^{*}$, 因此 $q$ 是实数。证毕。
					\paragraph*{定理2:属于不同本征值的本征函数是正交的\\}
						\par
						证明:假设
						\begin{equation}
						\hat{Q} f=q f, \quad \hat{Q} g=q^{\prime} g
						\end{equation}
						$\hat{Q}$ 是厄密算符。则有 $\langle f \mid \hat{Q} g\rangle=\langle\hat{Q} f \mid g\rangle$, 所以
						\begin{equation}
						q^{\prime}\langle f \mid g\rangle=q^{*}\langle f \mid g\rangle
						\end{equation}
						(再次, 内积是存在的因为假定本征函数是位于希耳伯特空间内)。但是 $q$ 是 实数 (由定理 1), 所以如果 $q^{\prime} \neq q$ 那么必然有 $\langle f \mid g\rangle=0$.

				补充一个公理:\textbf{可观测量算符的本征函数是完备的,(在希尔伯特空间中)任何函数都可以用它们的线性迭加来表达)}
			\subsubsection{连续谱}
		\subsection{广义统计诠释}	
			
















