%!TEX program = xelatex

%-----------------------------导言区---------------------------%
\documentclass[10pt,oneside,UTF8]{book}
\usepackage[fontset=mac]{ctex}
\usepackage[shortlabels]{enumitem}
\usepackage{graphicx,subfigure,booktabs,multirow,caption,setspace,listings,amsmath,amsfonts,lineno,multicol,float,stfloats}  % 直接导入常用包
\usepackage{xcolor,colortbl,rotating,bigstrut}
\usepackage{hyperref} % 生成书签和超链接
\usepackage{ulem} %解决文字下划线无法自动换行的问题(暂时无用)
\usepackage[hyperref=true,backend=biber,bibstyle=gb7714-2015,citestyle=numeric-comp,sorting=none,backref=true]{biblatex}
\usepackage{titlesec}
    %改变section、subsection里面字体的样式。中文黑体,英文TNR。
    \newfontfamily\sectionef{Times New Roman}
    \newcommand{\sectioncf}{\CJKfamily{FZHeiTi}}
    \titleformat*{\section}{\Large\bfseries\sectioncf\sectionef}
    \titleformat*{\subsection}{\large\bfseries\sectioncf\sectionef}
\usepackage{fancyhdr}
\usepackage{geometry} % 用于纸张格式的设置
\geometry{a4paper,scale=0.85,top=2cm,bottom=1.5cm,left=2.5cm,right=2.5cm}
    %定义公式编码格式
\numberwithin{equation}{section} %每次的NewSection都重置equation的ji shu
\renewcommand\theequation{\thesection.\arabic{equation}}
%----------------------标题与分栏线----------------------------%
\title{\fontsize{30pt}{30pt}\textbf{群论}}
% 对于空格有 \! \, \: \ \quad \qquad几种形式,间隔逐渐增大
\author
{\kaishu 郭蒙 \\
\kaishu 中山大学物理学院,广东,广州,510275} % \quad表示单个空格
\date{}

\setlength\columnsep{1cm} %设置分栏之后的栏间间距
\setlength{\columnseprule}{0.5pt} % 设置分栏之后的分割线宽度
%---------------------正文--------------------------%
\begin{document}
\maketitle  % 将前文的标题进行创建
\newpage
\pagenumbering{Roman}  %对目录使用罗马字体单独计算页数
\setcounter{page}{1}
\tableofcontents
\newpage
\setcounter{page}{1}	%重置对常规页面的阿拉伯级数
\pagenumbering{arabic}

\chapter{群的基本概念} %具体的笔记大纲暂时还无法确定 先以第一本教材 A.W.Joshi所著的大纲作为基本框架
    \section{群的定义}
    群是满足下列条件的元素的集合
    \begin{enumerate}
        \item 集合对群变换(composition)的封闭性,变换可以是加和也可以是乘积,集合中任意两元素的乘积也属于此集合\[A \circ B \in G , B \circ A \in G\]
        \item 群变换满足结合律(associative)\[A \circ (B \circ C)=(A \circ B) \circ C\]
        \item 集合中存在恒元(Identity element)满足\[E \circ A = A \circ E = A\]
        \item 集合中的任何元素都存在其相应的逆元素,满足\[A \circ A^{-1} =  A^{-1} \circ A =E\]
    \end{enumerate}

    这里需要特别说明的是,虽然一般情况下群使用的是乘法作为群封闭性的运算,但是也不排除个别情况下使用其他的运算法则作为群封闭的运算。例如,以加法作为某一个群的运算法则,其恒元就是零;若是以乘法作为某一个群的运算法则,其恒元就是一。在研究群的时候先搞清楚群的准确定义。

\section{一个例子:矩形对称群}
    对于一个矩形对称群,我们可以先标注出它相互对称八个点的编号,以便后面对变换进行排序。对于一个矩形的对称方式,我们有旋转和翻折两种形式的变换。

    先考虑旋转的情况,如果我们把矩形沿顺时针旋转$90^\circ$的变换记做$C_4$,那么相应的有$C_4^2$、$C_4^3$,对于$C_4^4$可直接用恒元E表示。


\end{document}