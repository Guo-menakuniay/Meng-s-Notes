\section{群的定义}
    群是满足下列条件的元素的集合
    \begin{enumerate}
        \item 集合对群变换(composition)的封闭性,变换可以是加和也可以是乘积,集合中任意两元素的乘积也属于此集合\[A \circ B \in G , B \circ A \in G\]
        \item 群变换满足结合律(associative)\[A \circ (B \circ C)=(A \circ B) \circ C\]
        \item 集合中存在恒元(Identity element)满足\[E \circ A = A \circ E = A\]
        \item 集合中的任何元素都存在其相应的逆元素,满足\[A \circ A^{-1} =  A^{-1} \circ A =E\]
    \end{enumerate}

    这里需要特别说明的是,虽然一般情况下群使用的是乘法作为群封闭性的运算,但是也不排除个别情况下使用其他的运算法则作为群封闭的运算。例如,以加法作为某一个群的运算法则,其恒元就是零;若是以乘法作为某一个群的运算法则,其恒元就是一。在研究群的时候先搞清楚群的准确定义。

\section{一个例子:矩形对称群}
    对于一个矩形对称群,我们可以先标注出它相互对称八个点的编号,以便后面对变换进行排序。对于一个矩形的对称方式,我们有旋转和翻折两种形式的变换。

    先考虑旋转的情况,如果我们把矩形沿顺时针旋转$90^\circ$的变换记做$C_4$,那么相应的有$C_4^2$、$C_4^3$,对于$C_4^4$可直接用恒元E表示。